



\section{Confidence Interval examples}



%--------------------
\subsection{Example}
A random sample of 15 observations is taken from a normally distributed population
of values. The sample mean is 94.2 and the sample variance is 24.86.
Calculate a 99\% confidence interval for the population mean.


\noindent \textbf{Solution}
$t_(14,0.005) = 2.977$
99\% CI is $94.2 \pm 2.977 \sqrt{24.86/15}$ \\i.e. $94.2 \pm 3.83$ \\i.e. $(90.37,98.03)$


\subsection{Example 1: paired T test}

\begin{center}
\begin{tabular}{|c|c|c|c|c|c|c|}
\hline
X & 5.20 & 5.15 & 5.17 & 5.16 & 5.19 & 5.15\\ \hline 
Y & 5.20 & 5.15 & 5.17 & 5.16 & 5.19 & 5.15\\
\hline
\end{tabular}
\end{center}

\subsection{Example 2}

Seven measurements of the pH of a buffer solution gave the
following results:

\begin{tabular}{|c|c|c|c|c|c|c|}
\hline
5.12 & 5.20 & 5.15 & 5.17 & 5.16 & 5.19 & 5.15\\
\hline
\end{tabular}

Task 1: Calculate the 95\% confidence limits for the true pH
utilizing $R$.


Solution. We are using Student t distribution with six degrees of
freedom and the following code gives us the confidence interval
for this problem.
\begin{verbatim}
>x <- c(5.12, 5.20, 5.15, 5.17, 5.16, 5.19, 5.15)
>n =length(x)
>alpha =0.05
>stderr =sd(x)/sqrt(n)
>LB=mean(x)+qt(alpha/2,6)* stderr
>UB=mean(x)+qt(1-alpha/2,6)* stderr
>LB
#[1] 5.137975
>UB
#[1]5.187739
\end{verbatim}

\subsection{Example 3} Ten replicate analyses of the concentration
of mercury in a sample of commercial gas condensate gave the
following results (in ng/ml) :

\begin{tabular}{|c|c|c|c|c|c|c|c|c|c|}
\hline
23.3 & 22.5 & 21.9 & 21.5 & 19.9 & 21.3 & 21.7 & 23.8 & 22.6 &
24.7\\
\hline
\end{tabular}

Compute 99\% confidence limits for the mean.
%http://www.stats.gla.ac.uk/steps/glossary/hypothesis_testing.html


%-----------------------
\subsection{Using Tables for the $t$-Distribution}

Degress of freedmom n-1

\begin{itemize}
\item For values between 31 and 40 we can use degrees freedom = 40

\item For samples sizes between 41 and 60, we can use degrees of freedom 60

\item For samples sizes between 61 and 120, we can use degrees of freedom 120

\item For samples larger than 120, we can use $\infty$
\end{itemize}

%------------------------------------------------------------------------------%

\noindent \textbf{Confidence Interval for a Mean (Small Sample)}
\begin{itemize}
\item The mean operating life for a random sample of $n = 10$ light bulbs is $\bar{x} = 4,000$ hours, with the sample
standard deviation $s = 200$ hours. \item The operating life of bulbs in general is assumed to be approximately normally distributed.\item
We estimate the mean operating life for the population of bulbs from which this sample was taken, using a 95 percent
confidence interval as follows:

\[4,000\pm(2.262)(63.3)  = (3857,4143)\]

\item The point estimate is 4,000 hours. The sample standard deviation is 200 hours, and the sample size is 10. Hence \[S.E(\bar{x} ) = { 200 \over \sqrt{10}} = 63.3\]

\item From last slide, the t quantile with $df=9$ is 2.262.
\end{itemize}


%------------------------------------------------------------------------%


% -- Lecture 8B
% -- Revise the Tables
% -- Sample Size Estimation for mean
% -- Example SSE for mean
% -- SSE for Proportion
% -- Example SSE for proportion
% -- Paired Test



\item Your company produces injected mouldings for rotary arms. The company claims that less than 2% of all product produced is defective. A random sample of 100 parts revealed that 4 parts were defective. 

Using a 5% level of significance is there evidence to close the line down? Justify your decision.
