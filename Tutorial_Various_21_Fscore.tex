
\documentclass[]{article}
\usepackage{framed}
\usepackage{amsmath}
\usepackage{amssymb}
\usepackage{multicol}
%opening

\begin{document}
\subsection*{Q4. Inference Procedures (Variant 2)}

\begin{itemize}
\item[(a)] \textbf{\textit{Binary Classification (4 Marks)}}\\
For following binary classification outcome table, calculate the following appraisal metrics.
\begin{itemize}	
\item 	accuracy;
\item 	recall;
\item 	precision;
\item  	F-measure.
\end{itemize}	

\begin{center}
\begin{tabular}{|c|c|c|}
\hline  & \phantom{spa}Predict Negative\phantom{spa} & \phantom{spa}Predict Positive\phantom{spa} \\ 
\hline\phantom{spa} Observed Negative \phantom{spa}&	9560	&	100	\\ 
\hline \phantom{spa}Observed Positive\phantom{spa} & 	270	&	70	\\ 
\hline 
\end{tabular} 
\end{center}

\item[(b)] \textbf{\textit{Theory of Statistical Inference (4 Marks)}}\\
\begin{itemize}
\item[i.](2 Marks) In the context of hypothesis testing, explain what a p-value is, and how it is used. Support your answer with a simple example.
\item[ii.](2 Marks) What is meant by Type I error and Type II error?
\end{itemize}

\item[(c)] \textbf{\textit{Single Sample Proportions (5 Marks)}}\\
\subsubsection*{Part B} %3 Marks
A well-known polling company estimates that $57\%$ of Irish voters support a new constitutional amendment. 800 people were randomly surveyed and asked about their voting preferences. 482 of the 800 people responded positively to the amendment. You are required to:

\begin{itemize}
\item [i.](1 Mark) Obtain a point estimate of the proportion of people supporting the constitutional amendment.
\item [ii.](4 Marks) Construct a 95\% confidence interval for the proportion of people in favour of the constitutional amendment.
\end{itemize}
