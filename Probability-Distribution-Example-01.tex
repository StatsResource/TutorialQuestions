\documentclass[a4paper,12pt]{article}
%%%%%%%%%%%%%%%%%%%%%%%%%%%%%%%%%%%%%%%%%%%%%%%%%%%%%%%%%%%%%%%%%%%%%%%%%%%%%%%%%%%%%%%%%%%%%%%%%%%%%%%%%%%%%%%%%%%%%%%%%%%%%%%%%%%%%%%%%%%%%%%%%%%%%%%%%%%%%%%%%%%%%%%%%%%%%%%%%%%%%%%%%%%%%%%%%%%%%%%%%%%%%%%%%%%%%%%%%%%%%%%%%%%%%%%%%%%%%%%%%%%%%%%%%%%%
\usepackage{eurosym}
\usepackage{vmargin}
\usepackage{amsmath}
\usepackage{multicol}
\usepackage{graphics}
\usepackage{epsfig}
\usepackage{enumerate}
\usepackage{framed}
\usepackage{subfigure}
\usepackage{fancyhdr}

\setcounter{MaxMatrixCols}{10}
%TCIDATA{OutputFilter=LATEX.DLL}
%TCIDATA{Version=5.00.0.2570}
%TCIDATA{<META NAME="SaveForMode" CONTENT="1">}
%TCIDATA{LastRevised=Wednesday, February 23, 2011 13:24:34}
%TCIDATA{<META NAME="GraphicsSave" CONTENT="32">}
%TCIDATA{Language=American English}

\pagestyle{fancy}
\setmarginsrb{20mm}{0mm}{20mm}{25mm}{12mm}{11mm}{0mm}{11mm}
\lhead{StatsResource} \rhead{Exponential Distribution}
\chead{Probability Distributions}
%\input{tcilatex}

\begin{document}
\large 
%------------------------------------------------------------------------------------%

\section{Review Question 8 :  Binomial Distribution}
Commuter trains have a probability 0.1 of delay
between Dublin and Maynooth. Supposing that the delays are all independent,
what is the probability that out of 10 journeys between Dublin and
Mullinar more than 8 do not have a delay.
\begin{itemize}
\item Reconsider the question : What is the probability that there is less than 2 delays.
\item $X$ is the variable for `delays', with Binomial parameters $n=10$, $p=0.1$
\item $P(X < 2) = P(X \leq 1) = P(X=0)+P(X=1)$
\item $P(X=0)$
\[P(X=0)= {10 \choose 0} \times 0.1^0  \times 0.9^10 = 0.34868\]
\item $P(X=1)$
\[P(X=1)= {10 \choose 1} \times 0.1^1  \times 0.9^9 = 0.38742\]
\item $P(X < 2) = 0.38742 + 0.34868 = 0.73610.$
\end{itemize}

%------------------------------------------------------------------------------------%

\section{Review Question 9 :  Poisson Distribution}
The number of visitors to a webserver per minute follows a Poisson
distribution. If the average number of visitors per minute is 4,
what is the probability that:
\begin{itemize}
\item[(i)] There are two or fewer visitors in one minute?;
\item[(ii)] There are exactly two visitors in 30 seconds?.
\end{itemize}



%------------------------------------------------------------------------------------%

\section{Review Question 9 :  Poisson Distribution}
Solution to Part 1
\begin{itemize}
\item $P(X\leq 2) = P(X = 0) + P(X = 1) + P(X = 2)$
\item $P(X = 0)$
 \[P(X = 0) = { e^{-4} \times 4^0\over 0!} = e^{-4}\]
\item $P(X = 1)$
\[P(X = 1) = { e^{-4} \times 4^1\over 1!} = 4e^{-4}\]
\item $P(X = 2)$
\[P(X = 2) = { e^{-4} \times 4^2\over 2!} = 8e^{-4}\]
\item Putting these values together
\[P(X\leq 2) = 13 \times e^{-4} = 0.238\]
\end{itemize}


%------------------------------------------------------------------------------------%


\documentclass[]{article}
\usepackage{framed}
\usepackage{amsmath}
\usepackage{amssymb}
\usepackage{multicol}
\usepackage{graphicx}
%opening

\begin{document}

%---------------------------------------%
\section*{Question 2 (Sample Variant 3)[25 marks]}
\begin{itemize}
%\subsection*{Part 2A : Poisson Distribution }

\item[(a)] \textbf{\textit{Probability Distributions (9 Marks)}}\\
Telephone calls arrive at a switchboard at the rate of 20 per hour.  Assume that the telecentre operators take 4 minutes to deal with a customer query.  Calculate the probability of :
\begin{itemize} 
\item[(i)] (3 Marks)                 3 or more calls arriving in any 4 minute period.
\item[(ii)] (2 Marks)               No phone calls arriving in a 4 minute period,
\item[(iii)] (3 Marks)              Exactly one phone call arriving in any 4 minute period,
\item[(iv)] (1 Marks)             What is the average and standard deviation of the number of phone calls arriving in a 4 minute period.
\end{itemize}
%\noindent (When answering, justify your answer with workings, or by reference to an axiom, theorem or rule.)


\bigskip
%-----------------------------------%
%\subsection*{Question 2B Binomial Distribution [3 Marks] } % 12 Marks
% New Question On Binomial
\item[(b)] \textbf{\textit{Probability Distributions (7 Marks)}}\\ On average, six people per hour use an electronic teller machine during the prime shopping hours in a department store. Therefore it is assumed that the expected time until the next customer will arrive will be 10 minutes. You may assume that the distributions of waiting times can be described by the exponential probability distribution.

\begin{itemize}
\item[(i)] (3 Marks) What is the probability that at least 10 minutes will pass between the arrival of two customers?
\item[(ii)] (2 Marks) What is the probability that after a customer leaves, another customer does not arrive for at least 20 minutes?
\item[(iii)] (2 Marks) What is the probability that a second customer arrives within 1 min after a first customer begins a banking transaction?
\end{itemize}
\bigskip
%-----------------------------------%
\item[(c)] \textbf{\textit{Probability Distributions (9 Marks)}}\\
For a digital communication channel, the probability of a bit being received in error is $5\%$. Consider the case where 100 bits are transmitted. Answer the following questions.

\begin{itemize}
\item[(i)] (3 marks)	What is the probability that the number of bits received in error is 5?
\item[(ii)] (3 marks) What is the probability that the number of bits received in error is greater than 10?
\item[(iii)] (3 marks) What is the expected value for the number of bit will be error. What is the variance for this value?
\end{itemize}

%\noindent(When answering, justify your answer with workings, or by reference to an axiom, theorem or rule.)




%\item[(d)] \textbf{\textit{Poisson Approximation of the Binomial Distribution (3 Marks)}}
%\begin{itemize}
%\item[(i)] (2 Marks) Describe how the Poisson distribution can be used to approximate the binomial distribution.
%\item[(ii)] (1 Mark) Explain the circumstances in which this approximation may be used in preference to the binomial distribution.
%\end{itemize}
\end{itemize}
\end{document}

\section{Review Question 9 :  Poisson Distribution}
Solution to Part 2
\begin{itemize}
\item The Poisson mean for 30 seconds is 2 occurrences.
\item Given m=2 : $P(X = 2)$
\[P(X = 2) = { e^{-2} \times 2^2\over 2!} = 0.2706 \]
\end{itemize}


%%%%%%%%%%%%%%%%%%%%%%%%%%%%%%%%%%%%%%%%%%%%%%%%%%%%%%%%%%
\section*{Question 2 (Sample Variant 2)[25 marks]}
\begin{itemize}

\item[(a)] \textbf{\textit{Probability Distribution (8 Marks)}}\\ % Exponential %6 MARKS
A computer software company which specializes in statistical software sells 4 software licences every day, on average. Answer the following questions.
    \begin{itemize}
    \item[i]  What is the probability that the software company sells at least two licence in one particular day?
    \item[ii] (3 marks) What is the probability that the software company will sell exactly one licence in one particular day?
    \item[iii] (3 marks) What is the probability that the software company will sell sixteen licences or more in a five day working week?
    \end{itemize}


\item[(b)] \textbf{\textit{Normal Distribution (9 Marks)}}\\ % Exponential %6 MARKS
 A telecommunications company has determined that the number of products sold on a weekly basis is normally distributed with a mean of 900 and standard deviation of 40.
\begin{itemize}
\item[i] (3 Mark) Estimate the proportion of weeks will the company sell more than 960 products.

\item[ii] (3 Mark) Estimate the proportion of weeks will the company sell less more than 860 products.
\item[iii] (3 Mark) Estimate the proportion of weeks will there be sales of between 880
products and 960 products?
\end{itemize}

\item[(c)] \textbf{\textit{Probability Distribution (8 Marks)}}\\ % Exponential %6 MARKS

For a digital communication channel, the probability of a bit being received in error is $10\%$. Consider the case where 100 bits are transmitted. Answer the following questions.

\begin{itemize}
\item[i] (3 marks)	What is the probability that the number of bits received in error is 10?
\item[ii]  What is the probability that the number of bits received in error is greater than 10?
\item[iii] 	What is the probability that the number of bits received in error does not exceed 20?
\end{itemize}

\end{itemize}

%------------------------------------------------------------------------------------%

\section{Review Question 10 :  Exponential Distribution}
Suppose that the average lifespan of a PC monitor is 5 years. You may assume that the lifespan of monitors follows an exponential probability distribution.
    \begin{enumerate}
    \item What is the probability that the lifespan of the monitor will be at least 5 years?
    \item What is the probability that the lifespan of the monitor will not exceed 4 years?
    \item What is the probability of the lifespan being between 5 years and 7 years?
    \end{enumerate}
(Note: In a previous version of this slide, the average lifespan was incorrectly given as 6 years).

%------------------------------------------------------------------------------------%

\section{Review Question 10 :  Exponential Distribution}
What is the probability that the lifespan of the monitor will be at least 5 years?
    \begin{enumerate}
    \item Using the following formulae
    \[P(X \leq k) = 1-e^{-k/\mu}\]
    \[P(X \geq k) = e^{-k/\mu}\]
    \item $P(X\geq 5)$
    \[P(X \geq 5) = e^{-5/5} = e^{-1} = 0.3678 \]
    \end{enumerate}


%------------------------------------------------------------------------------------%

\section{Review Question 10 :  Exponential Distribution}
What is the probability that the lifespan of the monitor will not exceed 4 years?
$P(X\leq 4)$
    \[P(X \leq 4) = 1- e^{-4/5} = 1 - e^{-0.8} = 0.5506 \]



%------------------------------------------------------------------------------------%

\section{Review Question 10 :  Exponential Distribution}
What is the probability of the lifespan being between 5 years and 7 years?
    \begin{enumerate}
    \item $P(5 \leq X \leq 7)$
    \item Too Low: $P(X \leq 5)$
    \[P(X \leq 5) = 1- e^{-5/5} = 1-e^{-1} = 0.6322 \]
    \item Too High: $P(X \geq 7)$
    \[P(X \geq 7) = e^{-7/5} = e^{-1.4} = 0.2465 \]
    \item P(Outside Interval) = P(Too High)+P(Too Low)
    \item P(Inside Interval) = 1 - P(Outside)
    \item $P(5 \leq X \leq 7)$ = 1- (0.6322+0.2465) = 0.12128
    \end{enumerate}