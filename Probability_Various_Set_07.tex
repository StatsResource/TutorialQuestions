
\subsection*{Question 6}

Let's assume that a sequence of bits (binary numbers) is transmitted and, at the other end, decoded; the decoder has a 10\% chance reading a bit incorrectly (i.e., reading a 0 as 1 or vice versa). Let $X$ be the number of errors in the sequence received (i.e., the decoded sequence). Calculate the probability that there are: \\[-0.2cm]

{\bf(a)} \emph{No} errors in a 20-bit string. \quad {\bf(b)} Less than three errors in a 10-bit string. \quad {\bf(c)} More than 10 errors in (i) a 50-bit string and (ii) a 100-bit string (hint: use tables). \quad {\bf(d)} Calculate the average number of errors in a 100-bit string. Calculate the standard deviation also.


\subsection*{Question 7}
We follow on from Question 6 but now consider the case where, to reduce the probability of error, each bit is sent \emph{three} times and then a ``majority vote'' approach is used to determine the value of each received bit. The following example explains the situation:\\[-0.5cm]
\begin{center}
	\begin{tabular}{ccccc}
		\hline
		&&&&\\[-0.3cm]
		\multirow{2}{*}{Sent} & $0$ & $1$ & $1$ & $0$ \\
		& $\overbrace{000}$ & $\overbrace{111}$ & $\overbrace{111}$ & $\overbrace{000}$ \\[0.2cm]
		\hline
		&&&&\\[-0.3cm]
		\multirow{2}{*}{Received} & $\underbrace{001}$ & $\underbrace{111}$ & $\underbrace{010}$ & $\underbrace{000}$ \\
		& $0$ & $1$ & $0$ & $0$ \\[0.2cm]
		\hline
		%\multicolumn{5}{c}{}
	\end{tabular}
\end{center}
$\Rightarrow$ there is one error in decoding the first $000$, but since the majority result is taken, this bit is correctly identified as a $0$. There are two errors in decoding the second $111$, so this bit is misread as a $0$. It is clear that a character is misread if the decoder makes \emph{two or three errors} in these blocks of three replicates.\\[-0.2cm]

{\bf(a)} Show that sending each bit 3 times reduces the error probability from 10\% to 2.8\%. \quad\\ {\bf(b)} Using this reduced value, $p=0.028$, calculate the probability that there are no errors in a 20-bit string. Compare this result to Q6(a). \quad {\bf(c)} Now assume that each bit is sent 5 times and, again, the majority vote approach is used. Calculate the probability that there are no errors in a 20-bit string in this case. %\quad {\bf(d)} Recalculate the two probabilities from part (c) using the Poisson approximation.
