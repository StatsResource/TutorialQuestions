\documentclass[a4paper,12pt]{article}
%%%%%%%%%%%%%%%%%%%%%%%%%%%%%%%%%%%%%%%%%%%%%%%%%%%%%%%%%%%%%%%%%%%%%%%%%%%%%%%%%%%%%%%%%%%%%%%%%%%%%%%%%%%%%%%%%%%%%%%%%%%%%%%%%%%%%%%%%%%%%%%%%%%%%%%%%%%%%%%%%%%%%%%%%%%%%%%%%%%%%%%%%%%%%%%%%%%%%%%%%%%%%%%%%%%%%%%%%%%%%%%%%%%%%%%%%%%%%%%%%%%%%%%%%%%%
\usepackage{eurosym}
\usepackage{vmargin}
\usepackage{amsmath}
\usepackage{framed}
\usepackage{graphics}
\usepackage{epsfig}
\usepackage{subfigure}
\usepackage{enumerate}
\usepackage{fancyhdr}

\setcounter{MaxMatrixCols}{10}
%TCIDATA{OutputFilter=LATEX.DLL}
%TCIDATA{Version=5.00.0.2570}
%TCIDATA{<META NAME="SaveForMode"CONTENT="1">}
%TCIDATA{LastRevised=Wednesday, February 23, 201113:24:34}
%TCIDATA{<META NAME="GraphicsSave" CONTENT="32">}
%TCIDATA{Language=American English}

\pagestyle{fancy}
\setmarginsrb{20mm}{0mm}{20mm}{25mm}{12mm}{11mm}{0mm}{11mm}
\lhead{StatsResource} 
\chead{Information Theory} \rhead{Exercises} %\input{tcilatex}
\begin{document}

\large 

\subsection*{Information}

\begin{framed}
\noindent The hartley (symbol Hart), also called a ban, or a dit (short for decimal digit), is a logarithmic unit which measures information or entropy, based on base 10 logarithms and powers of 10. \\
\noindent One hartley is the information content of an event if the probability of that event occurring is 1/10.\\
One Hart (i.e one ban) corresponds to $\ln(10)$ nat $= \log_2(10)$ bit or Sh, or approximately 2.303 nat, or 3.322 bit
\end{framed}

\noindent Entropy:
\[H(X) = \sum p(x_i) \left[\log_{10} \left({1 \over p(x_i)} \right)\right] \]  \bigskip 

{
\Large 
\begin{tabular}{|r||c|c|c|c|} \hline
$x_i$ & A & B & C & D \\ \hline \hline
$p(x_i)$ &0.5&0.25&0.125 &0.125 \\ \hline
$1/ p(x_i)$ &2&4&8 &8 \\ \hline
$\log_{10}[ 1/ p(x_i) ]$  &0.30103	 &0.60206 	  &0.90309&	0.90309 \\ \hline
$p(x_i) \times \mbox{log}[1/ p(x_i)]$ & 0.15051 &	0.15051&	0.11288 &	0.11288 \\ \hline
\end{tabular}
}
\\  \bigskip

$H(X) = 0.15051 +	0.15051+	0.11288 +	0.11288 $ \bigskip

$H(X) =  0.52680 \mbox{ Harts}$



\newpage 

\subsection*{Part (B)}

\begin{enumerate}[(i)]
\item The fair coin with, P(head)=0.50, P(tail)=0.50 


\begin{eqnarray*}
H &=& -1/2 \log_2(1/2) - 1/2\log_2(1/2) \\
&=& 1/2 + 1/2 \\
&=& 1 \mbox{bit} \\
\end{eqnarray*}

\item The biased coin, P(head)=0.75, P(tail)=0.25 

\begin{eqnarray*}
H &=& - 3/4 \log_2(3/4) - 1/4 \log_2(1/4) \\
&=& (3/4\times0.42) + 2/4 \\
&=& 0.31 + 0.5 \\
&=& 0.81 \mbox{ bits, approx.} \\
\end{eqnarray*}

\item A four symbol source , $P(A)=1/2$, $P(B)=1/4$, $P(C)=P(D)=1/8$ 
\begin{eqnarray*}
H &=& - 1/2 \log_2(1/2) - 1/4\log_2(1/4) - 1/8 \log_2(1/8) - 1/8 \log_2(1/8) \\
&=& - 1/2 \log_2(1/2) - 1/4\log_2(1/4) - 2/8 \log_2(1/8)  \\
&=& (- 1/2 \times -1) + (- 1/4  \times -2) + ( - 2/8  \times -3)  \\
&=& 1.75 \mbox{ bits} \\
\end{eqnarray*}

\end{enumerate}
\noindent bits = - $\log_2 p$  where p is the probability with which a particular value occurs

\begin{itemize}
\item  bits(A) $= - \log_2 1/2 = 1$
\item  bits(B) $= - \log_2 1/4 = 2$
\item  bits (C) = bits(D) $= - \log_2 1/8 = 3$
\end{itemize}

%---------------------------------------------------------------------------------%
\newpage
BLANK

\end{document}
