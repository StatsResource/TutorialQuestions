\newpage
Q5. The standard deviation of test scores obtained for a certain exam is 18 points. 
A random sample of 81 students has a sample mean of 70 points.

(a) State the point estimate for the mean score for all the students.
(b) Find the 95\% confidence interval for the average score for all students.
(c) Find the 99\% confidence interval for the average score for all students.

Q6. The amount spent (€’s) by customers in a shop are normally distributed. 
A random sample of 16 customers have these values:
\[19 21 35 29 12 35 7 18 21 14 29 20 12 24 32 23\]
(Sample  mean of €21.94 and a sample standard deviation of €8.40) 
Estimate a 95% confidence interval for the population mean.

Q7. The operating life of rechargeable cordless screwdrivers produced by a firm is assumed to 
the approximately normally distributed. A sample of 15 screwdrivers is tested and the mean 
life is found to be 8900 hours, with a sample standard deviation of 500 hours. 
Provide a 95% confidence interval for the population mean.


\newpage

%%%%%%%%%%%%%%%%%%%%%%%%%%%%%%%%%%%%%%%%%%%%%%%%%%%


Question 4.
A manufacturer of computer monitors  has, for many years, used a process giving a mean  working life of 4700 hours  for components.
A new process is tried to see if it will increase the life significantly. A sample of 100 monitors gave a mean life of 5000 hours, with a standard deviation of 1400 hours.
i.Compute a 95% confidence interval for the mean life of components built using the new process.
ii.Does this new process make a difference at the 5% level of significance? (Perform a two tailed test, then a one tailed test)

%%%%%%%%%%%%%%%%%%%%%%%%%%%%%%

\item \textbf{Worked Example 3} \\ Ten replicate analyses of the concentration
of mercury in a sample of commercial gas condensate gave the
following results (in ng/ml) :

\begin{tabular}{|c|c|c|c|c|c|c|c|c|c|}
\hline
23.3 & 22.5 & 21.9 & 21.5 & 19.9 & 21.3 & 21.7 & 23.8 & 22.6 &
24.7\\
\hline
\end{tabular}

%%%%%%%%%%%%%%%%%%%%%%%%%%%%%%%%%%%%%%%%%%%%%%%%%%%%%%%%%%%%%%%%%%%%%%%%%%%%%%%%%%%5


\item \textbf{Question 42. - Sample Size Estimation } % 10 Marks
Let $\pi$ be the proportion of workers in Ireland who spend at least one hour
per day in front of a computer terminal. Suppose that a researcher is going to take a
sample of n workers and estimate $\pi$ using $\hat{p}$, the proportion of workers in the sample
who spend at least one hour per day in front of a computer terminal.

\begin{itemize}
\item[a.] (1 mark) How large
should $n$ be if the researcher wants to be 90\% certain that his error is less than 0.01?
\end{itemize}
\end{enumerate}

\end{document}