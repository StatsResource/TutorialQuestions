	
	\item \textbf{Worked Example}
	%	Question 3.1 (f) \\ 
	A claim has been made that the mean body temperature of healthy adults is equal to 98.6? Fahrenheit. A sample of 106 people has produced a mean body temperature of 98.2? Fahrenheit and a standard deviation of 0.62. 
	
	(Hint: This is a test of whether the nominal mean is too high)
	\begin{itemize}
		\item[(i)] Clearly state your null and alternative hypotheses.
		\item[(ii)] Is this a One Tail or Two Tail test?
	\end{itemize}
	
	\item \textbf{Worked Example}
	%	Question 3.1 (b)\\
	A Health and Safety survey of 200 industrial accidents revealed that 53 
	were due to untidy working conditions.  
	
	A factory manager claims that the above sample is just a “one off”, and that less than 20\% of accidents are a result of untidy working conditions. 
	
	(Hint: This is a test of whether the nominal proportion is too low)
	\begin{itemize}
		\item[(i)] Clearly state your null and alternative hypotheses.
		\item[(ii)] Is this a One Tail or Two Tail test?
	\end{itemize}
	
	%%	\section{Single Sample Statistical Inference}
	
	
	\item \textbf{Worked Example   - Hypothesis Testing}	
	
	\begin{itemize}
		\item A sample of 50 households in one community
		shows that 10 of them are watching a TV special on the national
		economy. 
		\item In a second community, 15 of a random sample of 50
		households are watching the TV special. 
		\item We test the hypothesis
		that the overall proportion of viewers in the two communities does
		not differ, using the 1 percent level of significance, as follows:
	\end{itemize}
	
	\item \textbf{Worked Example   - Confidence Intervals}	
	
	A simple random sample is conducted of 1486 college students who are near completion of a bachelor’s degree in statistics. 802 students are female. Let $\pi$ be the proportion of female students who are near completion of their bachelor degree in `
	statistics.
	\begin{itemize}
		\item[a.] (1 mark) Provide a point estimate of $\pi$.
		\item[b.] (5 marks) Calculate a 99\% confidence interval for $\pi$.
		\item[c.] (9 marks) Use a 0.01 significance level to test the claim that the majority of
		college students who are near completion of a bachelor’s degree in statistics are
		female. [Clearly state the null and alternative hypotheses and your conclusion].\end{itemize}
	
	
	\item \textbf{Worked Example  - Single Sample t-Test (Small Sample)  } \\ % 10 Marks
	\begin{itemize}
		\item 	A web-based software company claims that the average amount of time it takes for
		online queries to be dealt with is less than 2 hours. 
		\item Out of a sample of 15 queries, the
		sample mean $\bar{x}$ = 3.5 hours and the standard deviation is 30 minutes.
	\end{itemize}
	
	\begin{itemize}
		\item[a.](2 marks) Construct the null and alternative hypothesis statements.
		\item[b.](2 marks) Test this claim using a significance level of 0.05.
		\item[c.](2 marks) Describe the two types of errors associated with hypothesis testing and how
		they relate to this question?
	\end{itemize}
	
	\item \textbf{Worked Example} \\ ABC Software has 125 programmers divided into two groups with 75 in
	Group A and 50 in Group B. In order to compare the efficiencies of the
	two groups, the programmers are observed for one day. \begin{itemize} \item The 75
		programmers of Group A averaged 76.21 lines of code with a standard
		deviation of 10.37. \item The 50 programmers of Group B averaged 72.72
		lines of code with a standard deviation of 10.07. \end{itemize}
	\begin{itemize}
		\item[a.](10 marks) Using a significance
		level of 5\%, test the hypothesis that there is no difference between the
		two groups versus the alternative that there is a difference. Clearly state
		your null and alternative hypotheses and your conclusion.
	\end{itemize}
	


\subsection{Example 2}

Seven measurements of the pH of a buffer solution gave the
following results:

\begin{center}
	\begin{tabular}{|c|c|c|c|c|c|c|}
		\hline
		5.12 & 5.20 & 5.15 & 5.17 & 5.16 & 5.19 & 5.15\\
		\hline
	\end{tabular}
\end{center}


Task 1: Calculate the 95\% confidence limits for the true pH
utilizing $R$.


Solution. We are using Student t distribution with six degrees of
freedom and the following code gives us the confidence interval
for this problem.
%%		\begin{verbatim}
%%		>x <- c(5.12, 5.20, 5.15, 5.17, 5.16, 5.19, 5.15)
%%		>n =length(x)
%%		>alpha =0.05
%%		>stderr =sd(x)/sqrt(n)
%%		>LB=mean(x)+qt(alpha/2,6)* stderr
%%		>UB=mean(x)+qt(1-alpha/2,6)* stderr
%%		>LB
%%		#[1] 5.137975
%%		>UB
%%		#[1]5.187739
%%		\end{verbatim}

\subsection{Example 3} Ten replicate analyses of the concentration
of mercury in a sample of commercial gas condensate gave the
following results (in ng/ml) :

\begin{center}
	\begin{tabular}{|c|c|c|c|c|c|c|c|c|c|}
		\hline
		23.3 & 22.5 & 21.9 & 21.5 & 19.9 & 21.3 & 21.7 & 23.8 & 22.6 &
		24.7\\
		\hline
	\end{tabular}
\end{center}

Compute 99\% confidence limits for the mean.
%http://www.stats.gla.ac.uk/steps/glossary/hypothesis_testing.html


\newpage
\section*{Question Set 2 : Confidence Intervals}
\begin{enumerate}
	
	\item \textbf{Worked Example}
	Calculate a 99\% confidence interval for the difference between the proportion of all Irish having access to the
	Internet and the proportion of all Spaniards having access to the internet.  (4 marks)
	
	
	
	\noindent \textbf{Standard Error for confidence interval}	
	
	\[\frac{p1(1 -p1)}{n1}+ \frac{p2(1 -p2)}{n2}\]	
	\[=\frac{0.750.25}{1000}+ \frac{0.700.30}{2000}     =  0.017103\]
	
	\noindent \textbf{Quantile for a 99\% confidence interval}
	\begin{itemize}
		\item 	significance level  =1\%
		\item	number of tails = 2
		\item	degrees of freedom = 
		\item	quantile = 2.576 
	\end{itemize}
	
	
	
	99\% Confidence Interval for difference of two proportions
	
	%================================================================= %
	
	
	Useful pieces of information
	
	
	Sample size  n=100
	
	
	\item \textbf{Worked Examples} \\
	
	The strength of concrete depends, to some extent, on the method used for drying. Two different methods showed the following results for independently tested specimens.  ( You may assume that there are equal variances).
	
	
	\begin{itemize}
		\item[(i)] Does Method 1 appear to produce concrete with a greater mean strength? State your conclusions clearly.
		\item[(ii)] Construct a 95\% confidence interval for the difference between the two means. Interpret this interval.
		
	\end{itemize}
	
	\item \textbf{Independent Sample Means hypothesis Test }
	%	Question 3.1 (d) \\
	
	\begin{itemize}
		\item 	A survey was carried out to investigate absenteeism in the building industry.
		\item Data on the number of sick days per year and type of job (unskilled or skilled)
		were collected for a random sample of employees.
		\item The mean number of sick days for 50 unskilled workers was 3.8 days with a standard deviation of 2.6 days. 
		\item The mean number of sick days for 60 skilled workers was 3 days with a standard deviation of 2.2 days. 
		\item (Hint: This is a test of whether that the means of the two groups are the same)
		
		\item Clearly state your null and alternative hypotheses.
		Is this a One Tail or Two Tail test?
	\end{itemize}
	
\end{enumerate}		







%------------------------------------------------------------- %

\subsection{Small Sample Test For Means - Worked Example}
%	D Hypothesis Testing - Example
The catering manager in a hotel suspects that the weight of loaves of bread delivered
daily by a bakery is consistently below the nominal weight of 800g. To test this,
10 loaves chosen at random from a day’s deliveries are weighed. The mean and
standard deviation of the ten weights are 792g and 25g, respectively.

\begin{enumerate}
	\item  Carry out a formal significance test.
	\item List the steps involved in this test 
	\item Calculate a 95\% confidence interval for the average weight of loaves produced
	\item Comment on the correspondence between the interval, as calculated, and the
	result of the test.
\end{enumerate}	



%		\noindent \textbf{Solution 2}
%		
%		\begin{itemize}
%			\item Confidence interval width is 3, so half-width is 1.5
%			
%			\item Seek n such that $1.96 \times \frac{9}{\sqrt{n}} = 1.5$
%			
%			\item Divide both sides by $1.96 \times 9$ \\
%			\[\frac{1}{\sqrt{n}} = \frac{1.5}{1.96 \times 9} =\]
%			
%			
%			\item invert and square boths sides.
%		\end{itemize}




\subsection{ Single Sample Proportion Test - Worked Examples}
%% Question 2 part b

A study of 1000 randomly chosen adults indicated that 450 had been to the cinema at least once in the previous year.

A cinema wants to test the hypothesis that 50\% of all Irish adults have been to the cinema in the last year.

Calculate the p-value for such a test and draw the appropriate conclusion.

Discussion: Based on this sample, we estimate the proportion to be 0.45  (i.e. 45\%)

\[ \mbox{Point Estimate} \hat{p} = 0.45\]

\noindent \textbf{Step A : Formally state the null and alternative hypotheses}

\begin{itemize}
	\item p : true proportion of Irish adults who have been to the cinema in the last year.
	
	\item	Null Hypothesis               Ho:p = 0.50        True proportion is 50%
	
	\item Alternative Hypothesis      Ha:p 0.50        True proportion is not 50%.
	
	
	\item	N.B. This is a two-tailed procedure.
\end{itemize}




\noindent \textbf{Step B : Compute the test statistic.}

Remember the general structure of a test statistic

TS =Observed Value-Null ValueStd. Error 



From the formulae

We have to compute the standard error for a proportion. 

( From formulae at back of exam paper)

S.E.(p) =p(1-p)n=0.450.551000= 0.0157




\noindent \textbf{Step 3: Calculate p-value}

P-value is found from Murdoch Barnes Tables 3 ( Normal distribution)

Absolute value  |-3.18| =3.18




\[ \mbox{P-Value} = P(Z \geq 3.18) = 0.00074\]


\noindent \textbf{Step 4: Interpret the p-value to make a decision.}

The significance level is 5\%.  The procedure is a two tailed test.


[ Black Board ]



