\newpage
\subsection*{Part 3. Confidence Interval ()}

Suppose that the mean weight of a sample of 16 items is 160g, and the sample standard deviation is 32g.



\begin{itemize}

 \item  Compute the standard error that would correspond to the sample mean.

 \item  State the appropropriate quantile from the $t-$distribution that would be used to compute the 95\% confidence interval for the mean.

 \item  Determine the 95\% confidence interval for the mean.
\end{itemize}
%%%%%%%%%%%%%%%%%%%%%%%%%%%%%%%%%%%%%%%%%%%%%%%%%%%%%%%%%%%%%%%%%%%%%%%%

\subsection*{Part 4. Hypothesis Testing ()}

\noindent A sample of 500 voters was taken by a political pollster to estimate the proportion of first preference votes a particular candidate will obtain in a forthcoming election. \\
\bigskip

\noindent It was found that 280 out of these 500 voters would give the candidate their first preference.
\[\hat{p} = 56\%  \mbox{    (i.e.  } 0.56)\]

\vspace{0.4cm}
\noindent
Using a significance level of 5\%, test the hypothesis that the percentage of voters who will give this particular candidate their first preference in the election is 60\%.\\


\begin{itemize}

 \item  Formally state the null and alternative hypotheses. (You may work on the basis that this is a two-tailed hypothesis test.)


 \item  Compute the Test Statistic for this hypothesis test.

 \item  Given that the critical value is 1.96, state your conclusion for this test.
\end{itemize}
\newpage



