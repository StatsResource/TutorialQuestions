
\item 
 During the early development of a medicated skin patch to help smokers break the habit, a test was conducted with 112 volunteers.  Because of the so-called
 "placebo effect", of people tending to respond positively just because attention is paid to them, about half the volunteers were given an unmedicated skin patch.  At the end of the study, the number of persons in each group who were abstinent and who were smoking are as follows:
 
 \begin{center}
 \begin{tabular}{|c|c|c|c|}
 \hline
 	&     Abstinent  &        Smoking   &     \\ \hline
 	
 	Medicated patch    &      21         &            36   &            57\\ \hline 
 	
 	Unmedicated patch  &       11       &              44   &            55\\ \hline 
 	
 	&       32       &              80    &          112\\ \hline
 \end{tabular} 
 \end{center}
 
 Let $\pi_1$ and $\pi_2$ denote the probabilities of quitting smoking with the medicated and unmedicated patches, respectively.
 
 \begin{itemize}
 	\item[(i)] Calculate point estimates for $\pi_1$ and $\pi_2$ 
 	
 	\item[(ii)] Compute the confidence interval of the difference in proportions $\pi_1 - \pi_2$ 
 	
 	\item[(iii)]Find a 95\% confidence interval for  .
 	\item[(iv)]  Based on the confidence interval found, can you conclude that (a) the success rate with the medicated patch is higher than for the control group that received the untreated patches?  (b) the medicated patch is not very effective?  Explain your answers.
 	\item  Carry out a testing procedure to investigate the claim that the medicated patch helps smokers break the habit.  State clearly the null hypothesis, alternative hypothesis, and conclusions drawn.
 \end{itemize}

%-------------------------------------------------------------------------------------------------- %

\item 
 During the early development of a medicated skin patch to help smokers break the habit, a test was conducted with 112 volunteers.  Because of the so-called
 "placebo effect", of people tending to respond positively just because attention is paid to them, about half the volunteers were given an unmedicated skin patch.  At the end of the study, the number of persons in each group who were abstinent and who were smoking are as follows:
 
 \begin{center}
 \begin{tabular}{|c|c|c|c|}
 \hline
 	&     Abstinent  &        Smoking   &     \\ \hline
 	
 	Medicated patch    &      21         &            36   &            57\\ \hline 
 	
 	Unmedicated patch  &       11       &              44   &            55\\ \hline 
 	
 	&       32       &              80    &          112\\ \hline
 \end{tabular} 
 \end{center}
 
 Let $\pi_1$ and $\pi_2$ denote the probabilities of quitting smoking with the medicated and unmedicated patches, respectively.
 
 \begin{itemize}
 	\item[(i)] Calculate point estimates for $\pi_1$ and $\pi_2$ 
 	
 	\item[(ii)] Compute the confidence interval of the difference in proportions $\pi_1 - \pi_2$ 
 	
 	\item[(iii)]Find a 95\% confidence interval for  .
 	\item[(iv)]  Based on the confidence interval found, can you conclude that (a) the success rate with the medicated patch is higher than for the control group that received the untreated patches?  (b) the medicated patch is not very effective?  Explain your answers.
 \end{itemize}

 
 Carry out a testing procedure to investigate the claim that the medicated patch helps smokers break the habit.  State clearly the null hypothesis, alternative hypothesis, and conclusions drawn.

