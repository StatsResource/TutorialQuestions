
\documentclass[a4paper,12pt]{article}
%%%%%%%%%%%%%%%%%%%%%%%%%%%%%%%%%%%%%%%%%%%%%%%%%%%%%%%%%%%%%%%%%%%%%%%%%%%%%%%%%%%%%%%%%%%%%%%%%%%%%%%%%%%%%%%%%%%%%%%%%%%%%%%%%%%%%%%%%%%%%%%%%%%%%%%%%%%%%%%%%%%%%%%%%%%%%%%%%%%%%%%%%%%%%%%%%%%%%%%%%%%%%%%%%%%%%%%%%%%%%%%%%%%%%%%%%%%%%%%%%%%%%%%%%%%%
\usepackage{eurosym}
\usepackage{vmargin}
\usepackage{amsmath}
\usepackage{graphics}
\usepackage{epsfig}
\usepackage{subfigure}
\usepackage{enumerate}
\usepackage{fancyhdr}
\usepackage{framed}

\setcounter{MaxMatrixCols}{10}
%TCIDATA{OutputFilter=LATEX.DLL}
%TCIDATA{Version=5.00.0.2570}
%TCIDATA{<META NAME="SaveForMode"CONTENT="1">}
%TCIDATA{LastRevised=Wednesday, February 23, 201113:24:34}
%TCIDATA{<META NAME="GraphicsSave" CONTENT="32">}
%TCIDATA{Language=American English}

\pagestyle{fancy}
\setmarginsrb{20mm}{0mm}{20mm}{25mm}{12mm}{11mm}{0mm}{11mm}
\lhead{StatsResource} \rhead{Worked Examples} \chead{Inference Procedures} %\input{tcilatex}

\begin{document}
\large 
\noindent ABC Software has 125 programmers divided into two groups with 75 in
Group A and 50 in Group B. In order to compare the efficiencies of the
two groups, the programmers are observed for one day. 
\begin{itemize} 
\item The 75
programmers of Group A averaged 76.21 lines of code with a standard
deviation of 10.37. 
\item The 50 programmers of Group B averaged 72.72
lines of code with a standard deviation of 10.07. 
\end{itemize}
\begin{enumerate}[(a)]
\item Using a significance
level of 5\%, test the hypothesis that there is no difference between the
two groups versus the alternative that there is a difference. Clearly state
your null and alternative hypotheses and your conclusion.
\end{enumerate}
\bigskip 

\begin{framed}
\noindent \textbf{Hypothesis Testing : Steps when using Critical Value approach}

\begin{itemize}
\item[1] Formally State the Null and Alternative Hypothesis \smallskip
{
\begin{itemize}
\item[$\ast$] \textbf{ALWAYS} include a short written description of both hypotheses.
\item[$\ast$] State hypotheses in mathematical notation where possible.

\end{itemize}
}
\item[2] Calculate Test Statistic (TS) using relevant formulas.\smallskip
\item[3] Determine the Critical Value (CV) from tables. \smallskip
\item[4] By comparing the values of the Test Statistic and Critical Value, decide whether to reject the Null Hypothesis.
\end{itemize}
\end{framed}
\newpage 
%%%%%%%%%%%%%%%%%%%%%%%%%%%%%%%%%%%%%%%%%%%%%%%%%%%%%%%
\subsection*{Statement of Hypotheses}
\noindent The first step in hypothesis testing is to specify the null hypothesis and an alternative hypothesis.
\begin{itemize}
    \item The Null Hypothesis ($H_0$) proposes that there is no difference in productivity, i.e. code output  between both groups of programmers.
    \item The Alternative Hypothesis $H_1$) proposes that there is a difference in productivity (There is no consideration about which group is better).
\end{itemize}
\begin{description}
\item[$H_0$:] $\mu_1 = \mu_2$ ( The population means are equal)
\item[$H_1$:] $\mu_1 \neq \mu_2$ (The population means are different)
\end{description}

\noindent Equivalently, we can write the hypotheses in terms of difference between groups.
\begin{description}
\item[$H_0$:] $\mu_1 - \mu_2 = 0$ 
\item[$H_1$:] $\mu_1 - \mu_2\neq   0$ 
\end{description}



\noindent (Remark: Two Tailed Test, therefore $k = 2$, and $\alpha = 0.05$)
%%%%%%%%%%%%%%%%%%%%%%%%%%%%%%%%%%%%%%%%%%%%%%%%%%%%%%%
\newpage
\subsection*{Computing the Test Statistic}


\begin{itemize}
    \item The point estimate is the difference in sample means:
    \[ \bar{x}_1 - \bar{x}_2 = 76.21 - 72.72 = 3.49\]
\end{itemize}
\noindent \textbf{Standard Errors}
\begin{itemize}
    \item The Standard Error Formula are typically provided in the exam paper, or an accompanying document.
    \item Each type of test has a different Standard Error Formula.
\end{itemize}

\begin{framed}
 The relevant formula for the standard error is
\[ S.E(\bar{x}_1 - \bar{x}_2) = \sqrt{{s^2_1\over n_1}+{s^2_2 \over n_2}} \]
\end{framed}
\bigskip


\begin{itemize}
\item $s^2_1$ and $s^2_1$ is the variance of samples $1$ and $2$ respectively.
\item $n_1$ and $n_2$ is the sample size of both samples.\bigskip

\end{itemize}
\begin{eqnarray*}
 S.E(\bar{x}_1 - \bar{x}_2) &=& \sqrt{{(10.37)^2 \over 75}+{(10.07)^2 \over 50}}   \\
&=& \sqrt{{107.5369 \over 75}+{101.4049 \over 50}}   \\
&=& \sqrt{1.4338 + 2.0280}   \\
&=& \sqrt{3.4618} \\
&=& 1.8605 \\
\end{eqnarray*}
\newpage 

\noindent \textbf{General Structure of the Test Statistic}
\begin{framed}
\[ TS = {\mbox{Point Estimate} - \mbox{Expected Parameter Value under $H_{0}$} \over \mbox{Std. Error}}\]
\end{framed}

\begin{eqnarray*}
TS &=& \frac{ 3.49 - 0 }{1.8605}\\
& & \\
&=& 1.8758\\
\end{eqnarray*}
%%%%%%%%%%%%%%%%%%%%%%%%%%%%%%%%%%%%%%%%%%%%%%%%%%%%%%%
\newpage 
\subsection*{Determine the Critical Value}

\begin{itemize}
\item The critical value is determined from Statistical Tables.
\item Determination depends on the significance level $\alpha$, the sample size $n$ and the number of tails in the test (i.e. is it a one-tailed or two tailed test.
\end{itemize}

\noindent \textbf{Significance Level}

\begin{itemize}
\item In hypothesis testing, the significance level $\alpha$ is the criterion used for rejecting the null hypothesis. 
%\item The significance level of a statistical hypothesis test is a fixed probability of wrongly rejecting the null hypothesis $H_0$, if it is in fact true.
%\item Equivalently, the significance level (denoted by $\alpha$) is the probability that the test statistics will fall into the \textbf{\emph{critical region}}, when the null hypothesis is actually true. 

\item In University courses, common choices for $\alpha$ are $0.05$ and $0.01$. We will use $\alpha =0.05$ (i.e. 5\%).
\item We have a large aggregate sample size ($n_1 + n2 > 30)$.
\item The hypothesis test is 2-tailed procedure.
\item The Critical Value is 1.96.
\end{itemize}




%%%%%%%%%%%%%%%%%%%%%%%%%%%%%%%%%%%%%%%%%%%%%%%%%%%%%%%
\newpage 
\subsection*{Decision}



\begin{framed}
\noindent \textbf{What are the conclusions}
\begin{itemize} 

\item[Yes:] We \textbf{reject the Null} Hypothesis. \\ \textit{We have sufficient evidence against Null Hypothesis.}

\item[No:] We \textbf{fail to reject} Null hypothesis. \\ \textit{We do not have sufficient evident against Null Hypothesis.}
\end{itemize}
{\normalsize
N.B. Note the terminology that we are using. Also note exactly what our conclusion is: We are talking about strength of evidence, rather than what is true or false.}

\end{framed}
\smallskip
\begin{framed}
\begin{itemize}
\item If $TS > CV$ , where $CV$ is a critical value corresponding to the sample size and confidence level, then reject the null hypothesis. 
\item  If $TS \leq CV$ , we fail to reject. null hypothesis. i.e. Not enough evidence. 
\end{itemize}
\end{framed}

\noindent Here $TS \leq CV$ , i.e. ($1.8758 \leq 1.96$), we fail to reject. null hypothesis. i.e. Not enough evidence of a difference. 



\end{document}
