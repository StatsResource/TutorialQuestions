
\subsection*{Question 3}
A sample of students from two universities was randomly selected. Each student had to complete the same programming task and the time to completion was recorded in each case. The results were as follows: \\

\begin{center}
\begin{tabular}{|c|c|c|}
\hline
&&\\[-0.4cm]
& University A & University B \\
\hline
&&\\[-0.4cm]
sample size & 15 & 15 \\
mean & 12.5\,\,\,hrs & 11.1\,\,\,hrs \\
variance & 3\,\,\,hrs$^2$ & 1.5\,\,\,hrs$^2$ \\
\hline
\multicolumn{3}{c}{}\\[-0.3cm]
\end{tabular}
\end{center}

We wish to test the hypothesis that there is no difference between universities at the 5\% level.\\[0.2cm]
\item State the null and alternative hypotheses. 
 \item If we do \emph{not} assume equal variances, what are the critical values? 
 \item Calculate the test statistic and, hence, provide your conclusion. 
 \item Between what two values does the p-value lie? (note: the p-value cannot be calculated exactly using the t-tables)



\subsection*{Question 4}
The government wish to know if there is a difference in the proportions of people living in rural and urban areas in support of a new policy. From a sample of 38 people in rural areas, it was found that 20 support the policy and from a sample of 116 individuals in urban areas, it was found that 70 support the policy. \\

\item Test the hypothesis that there is no difference in proportions at the 5\% level.

