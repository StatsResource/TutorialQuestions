
\subsection*{Example 3: Difference in Means (a) }
Two sets of patients are given courses of treatment under two different drugs. The benefits
derived from each drug can be stated numerically in terms of the recovery times; the readings are given below.

\begin{itemize}
\item Drug 1:  $n_1$ = 40 , $\bar{x}_1$ = 3.3 days and $s_1 = 1.524$
\item Drug 2:  $n_2$ = 45 , $\bar{x}_2$ = 4.3 days and $s_2 = 1.951 $
\end{itemize}
\end{frame}

%-------------------------------------------------------------------------------------------%
\begin{frame}
\subsection*{Example 3: Difference in Means (b) }
\begin{itemize}
\item
The first step in hypothesis testing is to specify the null hypothesis and an alternative hypothesis.
\item When testing differences between mean recovery times, the null hypothesis is that the two population means are equal.
\item That is, the null hypothesis is:\\
$H_0: \mu_1 = \mu_2$ ( The population means are equal)\\
$H_1: \mu_1 \neq \mu_2$ (The population means are different)\\
\end{itemize}
(Remark: Two Tailed Test, therefore $k = 2$, and $\alpha = 0.05$)


%-------------------------------------------------------------------------------------------%

\subsection*{Example 3: Difference in Means (c) }
\begin{itemize}
\item The observed difference in means is 1 day.
\item The relevant formula for the standard error is
\[ S.E(\bar{x}_1 - \bar{x}_2) = \sqrt{{s^2_1\over n_1}+{s^2_2 \over n_2}} \]
\item  \[ S.E(\bar{x}_1 - \bar{x}_2) = \sqrt{{(1.524)^2 \over 40}+{(1.951)^2 \over 45}}   \]
\item  \[ S.E(\bar{x}_1 - \bar{x}_2) = 0.377\mbox{ days}\]
\end{itemize}


%-------------------------------------------------------------------------------------------%
\begin{frame}[fragile]
\subsection*{Example 2: Difference in Means (d) }
\begin{itemize}
\item The Test statistic is therefore
\[ TS = {\mbox{observed} - \mbox{null} \over \mbox{Std. Error}}  = {1 - 0 \over 0.377 } = 2.65 \]
\item Lets compute the p-value of this : \\
p-value = $P(z \geq 2.65) = 0.0040$
\begin{verbatim}
> 1-pnorm(2.65)
[1] 0.004024589
\end{verbatim}

\item What is this value smaller than threshold $\alpha / k$? \\
\item $\alpha / k$ = $0.05/2$ = 0.0250? Yes the p-value is smaller than this.
\item \textbf{Conclusion:} we reject the null hypothesis. There is a significant different between both drugs, in terms of recovery times.

\end{itemize}

