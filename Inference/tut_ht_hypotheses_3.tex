\documentclass[a4paper,12pt]{article}
%%%%%%%%%%%%%%%%%%%%%%%%%%%%%%%%%%%%%%%%%%%%%%%%%%%%%%%%%%%%%%%%%%%%%%%%%%%%%%%%%%%%%%%%%%%%%%%%%%%%%%%%%%%%%%%%%%%%%%%%%%%%%%%%%%%%%%%%%%%%%%%%%%%%%%%%%%%%%%%%%%%%%%%%%%%%%%%%%%%%%%%%%%%%%%%%%%%%%%%%%%%%%%%%%%%%%%%%%%%%%%%%%%%%%%%%%%%%%%%%%%%%%%%%%%%%
\usepackage{eurosym}
\usepackage{vmargin}
\usepackage{amsmath}
\usepackage{framed}
\usepackage{graphics}
\usepackage{epsfig}
\usepackage{subfigure}
\usepackage{enumerate}
\usepackage{fancyhdr}

\setcounter{MaxMatrixCols}{10}
%TCIDATA{OutputFilter=LATEX.DLL}
%TCIDATA{Version=5.00.0.2570}
%TCIDATA{<META NAME="SaveForMode"CONTENT="1">}
%TCIDATA{LastRevised=Wednesday, February 23, 201113:24:34}
%TCIDATA{<META NAME="GraphicsSave" CONTENT="32">}
%TCIDATA{Language=American English}

\pagestyle{fancy}
\setmarginsrb{20mm}{0mm}{20mm}{25mm}{12mm}{11mm}{0mm}{11mm}
\lhead{Maths Resource} \rhead{Tutorial Sheet B} \chead{Hypothesis Testing} %\input{tcilatex}
\begin{document}
\section*{Hypothesis Testing : Tutorial Sheet C}
\begin{enumerate}

\item 
An IT competency test, used for staff recruitment, is devised so as to give a normal distribution of scores with a mean of 100. 
A random sample of 49 experienced IT users  who are given the test achieve a mean score of 121 with a standard deviation of 14. 

\begin{enumerate}[(a)]
\itemPerform a hypothesis test to assess whether this group of IT Users is unusual (i.e. have a different mean from the general population).
\itemCompute a 95\% confidence interval for the group.
\end{enumerate}
%%%%%%%%%%%%%%%%%%%%%%%%%%%%%%%%

 \begin{enumerate}
 \item  An IT competency test, used for staff recruitment, is devised so as to give a normal distribution of scores with a mean of 100. A random sample of 49 experienced IT users  who are given the test achieve a mean score of 121 with a standard deviation of 14.
 \begin{itemize}
  \item  Perform a hypothesis test to assess whether this group of IT Users is unusual (i.e. have a different mean from the general population).
  \item  Compute a 95\% confidence interval for the group.
 \end{itemize}
 