\documentclass[a4paper,12pt]{article}
%%%%%%%%%%%%%%%%%%%%%%%%%%%%%%%%%%%%%%%%%%%%%%%%%%%%%%%%%%%%%%%%%%%%%%%%%%%%%%%%%%%%%%%%%%%%%%%%%%%%%%%%%%%%%%%%%%%%%%%%%%%%%%%%%%%%%%%%%%%%%%%%%%%%%%%%%%%%%%%%%%%%%%%%%%%%%%%%%%%%%%%%%%%%%%%%%%%%%%%%%%%%%%%%%%%%%%%%%%%%%%%%%%%%%%%%%%%%%%%%%%%%%%%%%%%%
\usepackage{eurosym}
\usepackage{vmargin}
\usepackage{amsmath}
\usepackage{graphics}
\usepackage{framed}
\usepackage{epsfig}
\usepackage{subfigure}
\usepackage{enumerate}
\usepackage{fancyhdr}

\setcounter{MaxMatrixCols}{10}
%TCIDATA{OutputFilter=LATEX.DLL}
%TCIDATA{Version=5.00.0.2570}
%TCIDATA{<META NAME="SaveForMode"CONTENT="1">}
%TCIDATA{LastRevised=Wednesday, February 23, 201113:24:34}
%TCIDATA{<META NAME="GraphicsSave" CONTENT="32">}
%TCIDATA{Language=American English}

\pagestyle{fancy}
\setmarginsrb{20mm}{0mm}{20mm}{25mm}{12mm}{11mm}{0mm}{11mm}
\lhead{StatsResource} \rhead{Inference Procedures} \chead{Dixon Q-Test for Outliers} %\input{tcilatex}

\begin{document}
	\begin{framed}
		\noindent \textbf{Hypothesis Testing : Steps when using Critical Value approach}
		
		\begin{itemize}
			\item[1] Formally State the Null and Alternative Hypothesis \smallskip
			{
				\begin{itemize}
					\item[$\ast$] \textbf{ALWAYS} include a short written description of both hypotheses.
					\item[$\ast$] State hypotheses in mathematical notation where possible.
					
				\end{itemize}
			}
			\item[2] Calculate Test Statistic (TS) using relevant formulas.\smallskip
			\item[3] Determine the Critical Value (CV) from tables. \smallskip
			\item[4] By comparing the values of the Test Statistic and Critical Value, decide whether to reject the Null Hypothesis.
		\end{itemize}
	\end{framed}
	
	\noindent \textbf{Remarks}
	\begin{itemize} 
		\item Consider TS as a measure of the strength of evidence, and CV as a threshold that we have to cross for our evidence to be considered sufficiently strong.
		\item We usually use absolute value of Test Statistic $|TS|$.
		\item Key comparison
		\[ \mbox{ is } |TS| > CV \mbox{?} \]
	\end{itemize}	
	
	
	\noindent \textbf{What are the conclusions}
	\begin{itemize} 
		
		\item[Yes:] We \textbf{reject the Null} Hypothesis. \\ \textit{We have sufficient evidence against Null Hypothesis.}
		
		\item[No:] We \textbf{fail to reject} Null hypothesis. \\ \textit{We do not have sufficient evident against Null Hypothesis.}
	\end{itemize}	
	{	\normalsize
		N.B. Note the terminology that we are using. Also note exactly what our conclusion is: We are talking about strength of evidence, rather than what is true or false.}
	
	\newpage
	\section*{Outliers}
	
	
	\begin{itemize}
		\item In science, it is quite often the case that an outlier measurement is the result of faulty or unclean equipment, or a data entry error etc. 
		\item \textbf{Important:} Care must be take to assess that the measurement is an outlier, rather than an unusual result that is in fact genuine.
		\item It is good practice not to remove outliers from an overall analysis (\textit{not permanently anyway}). 
		\item However you may omit suspected outliers and run the analysis a second time, then present all of the obtained results, with and without the outliers.
		\item There may be an outlier, or multiple outliers, present in the data.
	\end{itemize}
	
	\noindent There are several formal hypothesis tests to determine presence of an outlier. The main one we will use is the Dixon Q-test.
	
	% \begin{itemize}
	%	\item The Dixon Test
	%	\item The Grubbs' Test
	% \end{itemize}
	
	\subsection*{Dixon Q Test for Outliers}
	\begin{itemize}
		\item The Dixon's Q test, or simply the Q test, is used for identification and rejection of outliers. 
		\item \textbf{(Important)} - This test is based upon the assumption of normality. 
		\item This test should be used sparingly and never more than once in a data set. 
		\item To apply a Q test for suspicious data, arrange the data in order of increasing values and calculate Q as defined:
		
		\[ Q = \frac{\text{gap}}{\text{range}} \]
		Where gap is the absolute difference between the outlier in question and the closest number to it. 
		
		\item 	If $Q_{Test} > Q_{CV}$ , where $Q_{CV}$ is a critical value corresponding to the sample size and confidence level, then reject the questionable data point. 
		\item Again, note that only one point may be rejected from a data set using a Q test.
		
	\end{itemize}
	
	
	
	\newpage
	
	
	
	
	\subsection*{Example}
	Consider the data set:
	\begin{framed}
		\[0.189,\ 0.167,\ 0.187,\ 0.183,\ 0.186,\]\[ 0.182,\ 0.181,\ 0.184,\ 0.181,\ 0.177 \,\]
	\end{framed}
	
	
	%==================================================================%
	
	\noindent \textbf{Step 1:} Hypotheses for the Dixon Test.
	
	\begin{framed}
		\begin{itemize}
			\item[$H_0$] No Outlier Present in Data
			\item[$H_1$] There is an Outlier present in Data (You may identify it
		\end{itemize}
	\end{framed}
	
	
	
	\noindent \textbf{Step 2:} Dixon Q Test Statistic\\ \smallskip
	To apply a Q test for suspicious data, arrange the data in order of increasing values and calculate Q as defined:
	\begin{framed}
		\[ Q = \frac{\text{gap}}{\text{range}} \]
	\end{framed}
	\noindent Where gap is the absolute difference between the outlier in question and the closest number to it. 
	
	%==================================================================%
	
	\medskip
	\noindent Consider the data set:
	\begin{framed}
		\[0.189,\ 0.167,\ 0.187,\ 0.183,\ 0.186,\]\[ 0.182,\ 0.181,\ 0.184,\ 0.181,\ 0.177 \,\]
	\end{framed}
	\noindent Now rearrange in increasing order:
	\begin{framed}
		\[0.167,\ 0.177,\ 0.181,\ 0.181,\ 0.182,\]\[ 0.183,\ 0.184,\ 0.186,\ 0.187,\ 0.189 \, \]
	\end{framed}
	
	%==================================================================%
	
	\noindent We hypothesize 0.167 is an outlier (based on it's large gap to next number, i.e. 0.010). 
	\begin{framed}
		\noindent Calculate The Test Statistic $Q_{Ts}$:
		{
			
			\[ Q_{TS}=\frac{\text{gap}}{\text{range}} = \frac{0.177-0.167}{0.189-0.167}=0.455.\]
		}
	\end{framed}
	

	
	
	
	%==================================================================%
\newpage 
	
	\textbf{Step 3:} Dixon Q Test Critical Value\\ \smallskip
{	
	\LARGE
	\begin{center}
%		\textbf{Critical Values for Dixon Q Test}
		\begin{tabular}{|c|c|c|c|}\hline
		\phantom{spa}N\phantom{spa}	&	$\alpha=0.10$	&	$\alpha=0.05$	&	$\alpha=0.01$	\\ \hline\hline
		3	&	0.941	&	0.97	&	0.994	\\ \hline
		4	&	0.765	&	0.829	&	0.926	\\ \hline
		5	&	0.642	&	0.71	&	0.821	\\ \hline
		6	&	0.56	&	0.625	&	0.74	\\ \hline
		7	&	0.507	&	0.568	&	0.68	\\ \hline
		8	&	0.468	&	0.526	&	0.634	\\ \hline
		9	&	0.437	&	0.493	&	0.598	\\ \hline
		10	&	0.412	&	0.466	&	0.568	\\ \hline
		11	&	0.392	&	0.444	&	0.542	\\ \hline
		12	&	0.376	&	0.426	&	0.522	\\ \hline
		13	&	0.361	&	0.41	&	0.503	\\ \hline
		14	&	0.349	&	0.396	&	0.488	\\ \hline
		15	&	0.338	&	0.384	&	0.475	\\ \hline
		16	&	0.329	&	0.374	&	0.463	\\ \hline
	\end{tabular}
\end{center}
}
\textit{Here: N is the sample size (\textit{it is usually denoted as $n$}).}

%==================================================================%



\begin{itemize}
	\item Choose the Critical Value based on sample size and significance level $\alpha$. \smallskip
	\item In this table we work on the basis of confidence level. \smallskip \item Let's use 95\% as our confidence level. \\ (i.e. 5\% significance, i.e. $\alpha=0.05)$ 
	
	%\item 	If $Q_{TS} > Q_{CV}$ , where $Q_{CV}$ is a critical value corresponding to the sample size and confidence level, then reject the null hypothesis. 
	%\item  If $Q_{TS} \leq Q_{CV}$ , we fail to reject. null hypothesis. i.e. Not enough evidence.
\end{itemize}

\medskip
\textbf{Step 4} : Dixon Q Test: Decison Rule
\begin{itemize}
	\item 	If $Q_{TS} > Q_{CV}$ , where $Q_{CV}$ is a critical value corresponding to the sample size and confidence level, then reject the null hypothesis. 
	\item  If $Q_{TS} \leq Q_{CV}$ , we fail to reject. null hypothesis. i.e. Not enough evidence. \smallskip
	%\item	With 10 observations and at 90\% confidence, $Q_{Test} = 0.455 > 0.412 =Q_{CV}$ , so we conclude 0.167 is an outlier.
	\item At 95\% confidence, $Q_{TS} \leq Q_{CV}$ i.e $ 0.455 \leq 0.466$ 
	\item Therefore we dont have enough evidence to classify the lowest value 0.167 as an outlier. 
	
	%\item This means that for this example we can be 90\% sure that 0.167 is an outlier, but we cannot be 95\% sure.
	%\bigskip
	%\item (Remark 95\% confidence is equivalent to 5\% signifificance)
\end{itemize}	

\newpage




\subsection*{Varying the Significance Level}
\begin{itemize}
	\item	With 10 observations and at 90\% confidence, $Q_{Test} = 0.455 > 0.412 =Q_{CV}$ , so we conclude 0.167 is an outlier.
	\item  However, at 95\% confidence, $Q_{Test} = 0.455 < 0.466$ = $Q_{CV}$ 0.167 is not considered an outlier. 
	
	\item This means that for this example we can be 90\% sure that 0.167 is an outlier, but we cannot be 95\% sure.
	\bigskip
	\item (Remark 95\% confidence is equivalent to 5\% signifificance)
\end{itemize}	


%=======================================================%

\newpage
BLANK
\end{document}	

