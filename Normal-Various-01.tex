%=====================================================%


Question 1(a)

Upper Limit ;  $U = \mu + 3 \sigma $
Lower Limit ;  $L = \mu - 3 \sigma $

Standardisation
Apply the standardisation formula	$Z=\frac{x-\mu}{\sigma} $	to both limits

\[ Z_U = \frac{U-\mu}{\sigma} =  \frac{(\mu + 3 \sigma)-\mu}{\sigma} = 3\]

Similary

$Z_l=-3$ 

\noindent \textbf{Probability of point being above Upper Limit}

From Murdoch Barnes Tables (page 13)  $P(Z \geq 3)=0.00135$

Probability of point being below Lower Limit


To find   we use the “Property of Symmetry”

“Property of Symmetry” -   for any value A

Therefore 

Conclusion: 
Probability of point being outside the 3 Sigma limits is

+ =0.00270 	(i.e. 0.27%)



Question 1(b)

Upper Limit ;  
LowerLimit ;  

Standardisation
Apply the standardisation formula	 	to both limits


Similary

\begin{itemize}
	\item 		Probability of point being above Upper Limit
	
	\item 	From Murdoch Barnes Tables (page 13)  
	
	\item 	Probability of point being below Lower Limit
	
	
	\item 	To find   we use the “Property of Symmetry”
	
	\item 	“Property of Symmetry” -   for any value A
	
	\item 	Therefore 
\end{itemize}	



Conclusion: 
Probability of point being outside the 3 Sigma limits is

+ =0.04550 	(i.e. 4.55\%) Question 1(b)
















Question 2C

Upper Limit ; 80.64		Mean		 	
Lower Limit ; 75.36		Standard Deviation	 

Standardisation
Apply the standardisation formula	 	to both limits


Similary



Probability of being above Upper Limit

From Murdoch Barnes Tables (page 13)  

Probability of being below Lower Limit


To find   we use the “Property of Symmetry”

“Property of Symmetry” -   for any value A

Therefore 

\noindent \textbf{Conclusion}\\
Probability of point being outside the specification limits 

+ is equal to

+ =0.2584 	(i.e. 26%)


%===========================================================%		
\bigskip		










Question 3A

Upper Limit ; 90		Mean		 	
Lower Limit ; 50		Standard Deviation	 

Standardisation
Apply the standardisation formula	 	to both limits


Similary



Probability of being above Upper Limit

From Murdoch Barnes Tables (page 13)  

Probability of being below Lower Limit


To find   we use the “Property of Symmetry”

“Property of Symmetry” -   for any value A

Therefore 

\noindent \textbf{Conclusion}\\
Probability of point being outside the specification limits is

+ is equal to


+ =0.01478  	(i.e. 1.5%)















Question 3B

Upper Limit ; 90		Mean		 	
Lower Limit ; 50		Standard Deviation	 

Standardisation
Apply the standardisation formula	 	to both limits


Similary



Probability of being above Upper Limit

From Murdoch Barnes Tables (page 13)  

Probability of being below Lower Limit


To find   we use the “Property of Symmetry”

“Property of Symmetry” -   for any value A

Therefore 

\noindent \textbf{Conclusion}\\
Probability of point being outside the specification limits is

+ is equal to

+ =0.01099  	(i.e. 1.1%)












\subsection{May 2012 Question 4 Normal Distribution / Theory}



%======================================================%
\noindent \textbf{Parameter Values}

Given the parameters of the normal distribution $X$ in the question.
\begin{itemize}
	\item Normal Mean $\mu = 73$ points
	\item Normal Standard Deviation $\sigma = 8$ points
\end{itemize}

\begin{itemize}
	\item $P(X \leq 91)$
	\item $P(65 \leq X \leq 89)$
\end{itemize}
Find the Z score for X = 91.

\[ Z = \frac{x- \mu}{ \sigma} = \frac{91 - 73}{8} =\frac{18}{8} = 2.25\]

Therefore we can say :\\ $P(X \leq 91)$ = $P(Z \leq 2.25)$ \\


From the tables $P(Z \leq 2.25) = 0.9877$
Therefore the probability of getting a grade lower than 91 is 0.9877 (i.e 98.77\%)


What is the probability of getting a score between 65 and 89.
Writing this mathematically:
\[ P(65 \leq X \leq 89) \]

%================================================================%
\begin{itemize}
	\item How many people get a score greater than 89? ($P(X\geq 89)$)
	\item How many people get a score less than 65? ($P(X\leq 65)$)
\end{itemize}

To compute $P(X \geq 89)$ first compute the Z-score.

\[ Z = \frac{x - \mu}{\sigma} = \frac{89 - 73}{8} =\frac{16}{8} = 2 \]

$P(X \geq 89)$ = $P(Z \geq 2)$ = 0.0225.

To compute $P(X \leq 65)$ first compute the Z-score.

\[ Z = \frac{x - \mu}{\sigma} = \frac{65 - 73}{8} =\frac{-8}{8} = -1 \]

$P(X \leq 65)$ = $P(Z \leq -1)$ 

\begin{itemize}
	\item We use the \textbf{symmetry rule}
	\[ P(Z \leq -1) = P(Z \geq +1) \]
	\item so we can say $P(X \leq 65)$ = $P(Z \geq +1)$ 
	\item From the statistical tables $P(Z \geq +1)$ = 0.1583.
\end{itemize}



\subsection{May 2012 Question 5 Normal Distribution }
Given
\begin{itemize}
	\item $X$ is the variable of interest.
	\item Normal Mean $\mu =25.5$ mpg
	\item Normal Standard Deviation $\sigma =4.5$ mpg
\end{itemize}

\begin{itemize}
	\item Find $x$ such that $P(X \geq x) = 0.30$
\end{itemize}



\subsection*{Solution}

From the Standard Normal Tables, find the value of $z$ that would give us
\[ P(Z \geq z) = 0.30 \]
Or if you are using the other type of tables 
\[ P(Z \leq z) = 0.70  \]
%-----------------------------------------------------%


\noindent \textbf{Normal Distribution}

\begin{center}
	\includegraphics[width=0.5\linewidth]{images/5BNormal1}
\end{center}



%============================================================================= %

\subsection{MA4413 Autumn 2008 paper (part 1)}
What proportion of retrieval times will be greater than 75 milliseconds?\\ \bigskip

\begin{itemize}
	\item Let X be the retrieval times, with $X \sim \mbox{N}(200,58^2)$.\\
	\item The first question asks us to find $P( X \geq 75)$. \\
	\item First compute the z score.
	\[ z_o =  {x_o - \mu \over \sigma} = {75 - 200 \over 58}  = -2.15 \]
\end{itemize}



\begin{center}
	\includegraphics[width=0.5\linewidth]{images/5BNormal2}
\end{center}

In this case, the probability of interest $P(X\geq 75)$, is represented by the white area under the curve.


{
	\noindent \textbf{MA4413 Autumn 2008 paper (part 1)}
	\begin{itemize}
		\item We can say
		\[ P( X \geq 75) = P( Z \geq -2.15)\]
		\item Using symmetry rule and complement rule
		\[ P( Z \geq -2.15) = P( Z \leq 2.15) = 1- P( Z \geq 2.15)\]
		\item From tables $P( Z \geq 2.15) = 0.0158$
		\item Therefore $P( Z \leq 2.15) = 0.9842$
		\item Furthermore $P( X \geq 75) = \boldsymbol{0.9842}$ [Answer].
	\end{itemize}
}

\noindent \textbf{Normal Distribution}

\begin{center}
	\includegraphics[width=0.5\linewidth]{images/5BNormal3}
\end{center}
