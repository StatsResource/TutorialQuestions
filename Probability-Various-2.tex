\documentclass[a4paper,12pt]{article}
%%%%%%%%%%%%%%%%%%%%%%%%%%%%%%%%%%%%%%%%%%%%%%%%%%%%%%%%%%%%%%%%%%%%%%%%%%%%%%%%%%%%%%%%%%%%%%%%%%%%%%%%%%%%%%%%%%%%%%%%%%%%%%%%%%%%%%%%%%%%%%%%%%%%%%%%%%%%%%%%%%%%%%%%%%%%%%%%%%%%%%%%%%%%%%%%%%%%%%%%%%%%%%%%%%%%%%%%%%%%%%%%%%%%%%%%%%%%%%%%%%%%%%%%%%%%
\usepackage{eurosym}
\usepackage{vmargin}
\usepackage{amsmath}
\usepackage{framed}
\usepackage{multicol}
\usepackage{graphics}
\usepackage{epsfig}
\usepackage{subfigure}
\usepackage{enumerate}
\usepackage{fancyhdr}

\setcounter{MaxMatrixCols}{10}
%TCIDATA{OutputFilter=LATEX.DLL}
%TCIDATA{Version=5.00.0.2570}
%TCIDATA{<META NAME="SaveForMode"CONTENT="1">}
%TCIDATA{LastRevised=Wednesday, February 23, 201113:24:34}
%TCIDATA{<META NAME="GraphicsSave" CONTENT="32">}
%TCIDATA{Language=American English}

\pagestyle{fancy}
\setmarginsrb{20mm}{0mm}{20mm}{25mm}{12mm}{11mm}{0mm}{11mm}
\lhead{MathsResource} \chead{Introduction to Probability} \rhead{Tutorial Sheet 2} %\input{tcilatex}
\begin{document}
\begin{enumerate}

\item The following contingency table illustrates the number of 500 students in different
departments according to gender.

\begin{center}
\begin{tabular}{|c|c|c|c|}
  \hline
  % after \\: \hline or \cline{col1-col2} \cline{col3-col4} ...
   & Computer Science & Engineering & Food Science \\\hline
  Males & 180 & 100 & 20  \\  \hline
  Females & 60 & 80 & 60  \\ \hline

  \hline
\end{tabular}
\end{center}

\begin{enumerate}[(i)]
\item (1 Mark) What is the probability that a randomly chosen person from the sample is a
computer science student?
\item (1 Mark) What is the probability that a randomly chosen person from the sample is both female and studying statistics?
% \item (1 Mark) What is the probability that a randomly chosen person from the sample is male?
\item (1 Mark) Given that a student studies statistics, what is the probability that the student is female?
%\item (1 Mark) What is the probability that a randomly chosen person from the sample is a male or a statistics student?
% \item (2 marks) Given that the student is female, what is the probability that she is an equine science student?
\end{enumerate}



\item 
 An urn contains 10 disks, 6 white and 4 red.  Two disks are selected, without replacement, from the urn.  Calculate the following probabilities: 
\begin{enumerate}[(a)]
 \item Exactly one disk chosen is white. 
 \item At least one disk is white. 
 \item Neither disk chosen is red 
 \item At most one disk chosen is red 
 \end{enumerate} 
%=========================%

\item An IT consultant is responsible for three software engineering projects X, Y and Z.
They know that the probability of completing project X in time is 0.99, for project Y this probability is 0.95
and for project Z it is 0.80.

\begin{enumerate}[(a)]
\item What assumption do you need to make in order to calculate the probability
of completing all three projects in time, from the information given?
\item Calculate the probability of completing all three projects in time.
\item Calculate the probability that only projects X and Y will be completed on time.
\end{enumerate}

    \item 
An electronics assembly subcontractor receives its entire supply of resistors from two suppliers. Company A provides 65\% of the subcontractor's resistors, while company B supplies the remainder. The additional information has also been made available.
\begin{itemize}
\item 4\% of the resistors provided by company A failed the final test,
\item 5\% of company B's resistors also fail the final test.
\end{itemize}
\noindent Answer the following questions:
\begin{enumerate}[(a)]
\item What is the probability that a resistor fails the final test?
\item  What is the probability that a resistor fails the final test given that the resistor in question came from company A?
\item  What is the probability that a resistor that fails final test was supplied by company A?
\end{enumerate}
