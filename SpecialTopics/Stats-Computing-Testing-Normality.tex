

%-------------------------------------------------%
\begin{frame}[fragile]
\frametitle{Graphical Procedures for assessing Normality}

\begin{itemize}
\item The normal probability (Q-Q) plot is a very useful tool for determining whether or not a data set is normally distributed.
\item Interpretation is simple. If the points follow the trendline (provided by the second line of \texttt{R} code \texttt{qqline}).
\item One should expect minor deviations. Numerous major deviations would lead the analyst to conclude that the data set is not normally distributed.
\item The Q-Q plot is best used in conjunction with a formal procedure such as the Shapiro-Wilk test.
\end{itemize}

\begin{verbatim}
>qqnorm(CWdiff)
>qqline(CWdiff)
\end{verbatim}

\end{frame}

%-------------------------------------------------%

\begin{frame}
\frametitle{Graphical Procedures for Assessing Normality}

\begin{center}
\includegraphics[scale=0.32]{10AQQplot}
\end{center}
\end{frame}
