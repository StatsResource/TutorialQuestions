%=================================================%
\newpage
\subsection{Binary Classification}
\noindent \textbf{Defining True/False Positives}
In general, Positive = identified and negative = rejected. Therefore:

\begin{itemize}
\item True positive = correctly identified

\item False positive = incorrectly identified

\item True negative = correctly rejected

\item False negative = incorrectly rejected
\end{itemize}
\subsubsection*{Medical Testing Example:}
\begin{itemize}
\item True positive = Sick people correctly diagnosed as sick

\item False positive= Healthy people incorrectly identified as sick

\item True negative = Healthy people correctly identified as healthy

\item False negative = Sick people incorrectly identified as healthy.
\end{itemize}
%-------------------------------------------------- %
\newpage
\subsection{Definitions (From Week 1)}
\textbf{Confusion Matrix} \\
The confusion
table is a table in which the rows are the observed categories of the dependent and
the columns are the predicted categories. When prediction is perfect all cases will lie on the
diagonal. The percentage of cases on the diagonal is the percentage of correct classifications. 

\textbf{Accuracy Rate}\\
The accuracy rate calculates the proportion ofobservations being allocated to the \textbf{correct} group by the predictive model. It is calculated as follows:
\[ \frac{
\mbox{Number of Correct Classifications }}{\mbox{Total Number of Classifications }} \]

\[ = \frac{TP + TN}{TP+FP+TN+FN}\]

\medskip

\noindent \textbf{Misclassification Rate}\\
The misclassification rate calculates the proportion ofobservations being allocated to the \textbf{incorrect} group by the predictive model. It is calculated as follows:
\[ \frac{
\mbox{Number of Incorrect Classifications }}{\mbox{Total Number of Classifications }} \]

\[ = \frac{FP + FN}{TP+FP+TN+FN}\]
\newpage
%---------------------------------------------------------- %
\newpage
\section{Model Accuracy}
%http://www.unt.edu/rss/class/Jon/Benchmarks/CrossValidation1_JDS_May2011.pdf
Prediction error refers to the discrepancy or difference between a predicted value (based on a
model) and the actual value. In the standard regression situation, prediction error refers to how
well our regression equation predicts the outcome variable scores of new cases based on
applying the model (coefficients) to the new cases’ predictor variable scores.

\begin{equation}
\text{Accuracy}=\frac{TP+TN}{TP+FP+FN+TN}
\end{equation}

\begin{equation}
\text{Precision}=\frac{TP}{TP+FP} \, 
\end{equation}

\begin{equation}
\text{Recall}=\frac{TP}{TP+FN} \, 
\end{equation}

\subsection{Misclassification Cost}
\begin{itemize}
\item As in all statistical procedures it is helpful to use diagnostic procedures to assess the efficacy of the analysis. We use \textbf{cross-validation} to assess the classification probability.
Typically you are going to have some prior rule as to what is an \textbf{acceptable misclassification rate}.

\itemThose rules might involve things like, \textit{``what is the cost of misclassification?"} Consider a medical study where you might be able to diagnose cancer.

\item There are really two alternative costs. 
\begin{itemize}
\item[$\ast$] The cost of misclassifying someone as having cancer when they don't.
This could cause a certain amount of emotional grief. Additionally there would be the substantial cost of unnecessary treatment.

\item[$\ast$] There is also the alternative cost of misclassifying someone as not having cancer when in fact they do have it.
\end{itemize}
\item A good classification procedure should
\begin{itemize}
\item[$\ast$] result in few misclassifications
\item[$\ast$] take \textbf{\textit{prior probabilities of occurrence}} into account
\item[$\ast$] consider the cost of misclassification
\end{itemize}

\item For example, suppose there tend to be more financially sound firms than bankrupt
firm. If we really believe that the prior probability of a financially
distressed and ultimately bankrupted firm is very small, then one should
classify a randomly selected firm as non-bankrupt unless the data
overwhelmingly favor bankruptcy.



\item There are two costs associated with discriminant analysis classification: The true misclassification cost per class, and the expected misclassification cost (ECM) per observation.

\item \textbf{Example} Suppose there we have a binary classification system, with two classes: class 1 and class 2.
Suppose that classifying a class 1 object as belonging to class 2 represents a more serious error than classifying a class 2 object as belonging to class 1. There would an assignable cost to each error.
\textbf{c(i$|$j)} is the cost of classifying an observation into class $j$ if its true class is $i$.
The costs of misclassification can be defined by a cost matrix.

\begin{center}
\begin{tabular}{|c|c|c|}
\hline
% after \\: \hline or \cline{col1-col2} \cline{col3-col4} ...
& Predicted & Predicted \\
& Class 1 & Class 2 \\  \hline
Class 1 & 0 & $c(2|1)$  \\ \hline
Class 2 & $c(1|2)$ & 0 \\
\hline
\end{tabular}
\end{center}

\end{itemize}

\noindent \textbf{Expected cost of misclassification (ECM)}
\begin{itemize}
\item Let $p_1$ and $p_2$ be the prior probability of class 1 and class 2 respectively.
Necessarily $p_1$ + $p_2$ = 1.

\itemThe conditional probability of classifying an object as class 1 when it is in fact from
class 2 is denoted $p(1|2)$.
\item Similarly the conditional probability of classifying an object as class 2 when it is in
fact from class 1 is denoted $p(2|1)$.

\[ECM = c(2|1)p(2|1)p_1 + c(1|2)p(1|2)p_2\]
\textit{(In other words: the sum of the cost of misclassification times the (joint) probability of that misclassification.)}

\item A reasonable classification rule should have ECM as small as possible.
\end{itemize}




\end{document}
