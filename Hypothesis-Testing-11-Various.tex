\item 
The height of 100 Americans and 50 Spaniards was observed. The mean and
standard deviation of the height of the Americans was 172cm and 13cm,
respectively. The mean and standard deviation of the height of the Spaniards
was 167cm and 12cm, respectively.

\begin{itemize}
	\item[(i)]Calculate a 95\% confidence interval for the difference between the mean height
	of all Americans and the mean height of all Spaniards.
	
	
	\item[(ii)] Without doing any further calculations, test the hypothesis that the mean
	height of all Americans is equal to the mean height of all Spaniards. Give a brief
	justification of your conclusion. What is the significance level of this test'?
\end{itemize}

\item 
The mean and standard deviation of the salaries of 16 Irish full-time workers are 5000 and 3000 Euros, respectively.
\begin{itemize}
	\item[(i)] Test the hypothesis that the mean salary of all Irish full-time workers is E4000 at a significance level of 5\%.
	\item[(ii)] What assumption is made in this testing procedure? Is this assumption reasonable?
\end{itemize}

\item (Not using for MS4222 2018).\\
A survey of 1000 Irish indicates that 750 have access to the Internet. A survey of 2000 Spaniards
indicates that 1400 have access to the Internet.
\begin{itemize}
	\item[(i)]  By calculating the appropriate p-value, test the hypothesis that the proportion of all Irish
	having access to the Internet is equal to the proportion of all Spaniards having access to the
	internet at a significance level of 5%.
	
	\item[(ii)] Calculate a 99\% confidence interval for the difference between the proportion of all Irish
	having access to the Internet and the proportion of all Spaniards having access to the
	internet.
\end{itemize}

\item (Not using for MS4222 2018).\\
Let $\pi$ be the proportion of workers in Ireland who spend at least one hour
per day in front of a computer terminal. Suppose that a researcher is going to take a
sample of n workers and estimate $\pi$ using $\hat{p}$, the proportion of workers in the sample
who spend at least one hour per day in front of a computer terminal.

\begin{itemize}
	\item[a.] (1 mark) How large
	should $n$ be if the researcher wants to be 90\% certain that his error is less than 0.01?
\end{itemize}
\end{enumerate}
\end{document}