\documentclass[a4paper,12pt]{article}
%%%%%%%%%%%%%%%%%%%%%%%%%%%%%%%%%%%%%%%%%%%%%%%%%%%%%%%%%%%%%%%%%%%%%%%%%%%%%%%%%%%%%%%%%%%%%%%%%%%%%%%%%%%%%%%%%%%%%%%%%%%%%%%%%%%%%%%%%%%%%%%%%%%%%%%%%%%%%%%%%%%%%%%%%%%%%%%%%%%%%%%%%%%%%%%%%%%%%%%%%%%%%%%%%%%%%%%%%%%%%%%%%%%%%%%%%%%%%%%%%%%%%%%%%%%%
\usepackage{eurosym}
\usepackage{vmargin}
\usepackage{framed}
\usepackage{amsmath}
\usepackage{graphics}
\usepackage{epsfig}
\usepackage{subfigure}
\usepackage{enumerate}
\usepackage{fancyhdr}

\setcounter{MaxMatrixCols}{10}
%TCIDATA{OutputFilter=LATEX.DLL}
%TCIDATA{Version=5.00.0.2570}
%TCIDATA{<META NAME="SaveForMode"CONTENT="1">}
%TCIDATA{LastRevised=Wednesday, February 23, 201113:24:34}
%TCIDATA{<META NAME="GraphicsSave" CONTENT="32">}
%TCIDATA{Language=American English}

\pagestyle{fancy}
\setmarginsrb{20mm}{0mm}{20mm}{25mm}{12mm}{11mm}{0mm}{11mm}
\lhead{Maths Resource} \rhead{Tutorial Sheet} \chead{Probability} %\input{tcilatex}

\begin{document}
\begin{enumerate}
\item I pick 3 cards from a pack of 52. Calculate the probability that 
i)I pick exactly one spade
ii)I pick at least one spade
iii)I pick exactly one spade, given that I pick at least one spade.

%%%%%%%%%%%%%%%%%%%%%%%%%%%%%%%%%%%%%%%%%%%%%%%%%%%%%%

\itemA coin is thrown 4 times. Calculate 
i)The probability of throwing 4 heads
ii)The probability of throwing 3 heads 
iii)The probability of throwing 3 heads, given that the result of the first roll is tails.


\item A machine is composed of 3 components, which function independently of each other with probabilities p1, p2 and p3, respectively. Calculate the probability that the machine works when
a)the machine only works when all the components are working
b)the machine works when at least one of the components works.

\item  A die is thrown twice. A is the event that the sum is 7. B is the event that the first die roll results in a 1. C is the event that the second die roll results in a 6. 
i)Are the events A, B and C independent?
ii)Are the events A, B and C pairwise independent?
\item A coin is tossed until it falls on the same side twice in a row.
\begin{itemize}
\item[(i)] Define the set of elementary events of such an experiment.
\item[(ii)]  Calculate the probability that the coin is thrown exactly 5 times.
\item[(iii)]  Calculate the probability that the number of throws is even.
\end{itemize}
%--------------------------------------------------%
\itemIn a newspaper on average 1 in 10 000 characters is incorrectly printed. Suppose the paper contains 50 000 characters. Calculate the exact probability that 
 \begin{itemize}
\item[(i)]no printing errors are made
\item[(ii)]at least 3 errors are made
 \end{itemize}
Using the appropriate approximation, estimate these two probabilities. 
%--------------------------------------------------%

\item Suppose X has the following cumulative distribution function. F(x)=0, for x≤0, 
F(x)=1, for x≥5 and F(x) = x3/125 for 0≤ x ≤ 5. Derive E(X) and Var(X).



\item  Suppose calls come into a call centre randomly at a rate of one per 30 seconds.
i) What is the distribution of the time to the second call?
ii) Using this distribution, calculate the probability that the second call arrives within a minute. 
iii) Using the appropriate discrete distribution, calculate the probability that at least 2 calls are received in a minute (note this probability has to be the same as above).
iv) What is the exact distribution of the time to the 200th call?
v) Using the central limit theorem, give the normal distribution which approximates the distribution from iv).
vi) Using your answer from v), estimate the probability that the time to the 200th call is less than 102 minutes.

