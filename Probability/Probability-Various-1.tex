%=========================%

A doctor treating a patient issues a prescription for antibiotics and provides for two repeat prescriptions. 
\begin{itemize}
    \item The probability that the infection will be cleared by the first prescription is $p_1$ =0.6.
\item The probability that successive treatments are successful, given that previous prescriptions were not successful are $p_2$ = 0.5, $p_3$ = 0.4.
\end{itemize} Calculate the probability that:

\begin{enumerate}[(a)]
\item a patient will require the third prescription,
\item the patient is still infected after the third prescription,
\item the patient is cured by the third prescription, given that the patient is eventually cured.
\end{enumerate}

%============================%


 A new test has been developed to diagnose a particular disease. 
If a person has the disease, the test has a 98\% chance of identifying them as having the 
disease. If a person does not have the disease, the test has a 5\% chance of identifying them as having the disease. 5\% of the population have this disease. 
Suppose we select a person at random from the population.

\begin{enumerate}[(a)]
\item What is the probability that the test will identify them as having the disease?
\item What is the probability that the person has the disease given that the test identifies 
them as having the disease?
\end{enumerate}



\end{enumerate}



\end{document}