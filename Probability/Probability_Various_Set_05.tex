
%============================%



\subsection*{Question 6}
By applying \emph{Little's Law}, answer the following questions:\\[-0.2cm]
\begin{itemize}
\item[(a)] A section of road takes on average 5 minutes to negotiate. Cars arrive to this section at a rate of 2 per minute. On average, how many cars are on the road?   \item[(b)] Jobs are sent to a supercomputer at a rate of 4 per hour. On average we wait 30 minutes from the time of sending to the time of completion. How many jobs are in the system on average?  \item[(c)] On average 20 customers arrive to a cafe per hour. If there are 10 people in the cafe on average, how long do they spend there?
\end{itemize}


\subsection*{Question 7}
We follow on from Question 6 but now consider the case where, to reduce the probability of error, each bit is sent \emph{three} times and then a ``majority vote'' approach is used to determine the value of each received bit. The following example explains the situation:\\[-0.5cm]
\begin{center}
\begin{tabular}{ccccc}
\hline
&&&&\\[-0.3cm]
\multirow{2}{*}{Sent} & $0$ & $1$ & $1$ & $0$ \\
& $\overbrace{000}$ & $\overbrace{111}$ & $\overbrace{111}$ & $\overbrace{000}$ \\[0.2cm]
\hline
&&&&\\[-0.3cm]
\multirow{2}{*}{Received} & $\underbrace{001}$ & $\underbrace{111}$ & $\underbrace{010}$ & $\underbrace{000}$ \\
& $0$ & $1$ & $0$ & $0$ \\[0.2cm]
\hline
%\multicolumn{5}{c}{}
\end{tabular}
\end{center}
$\Rightarrow$ there is one error in decoding the first $000$, but since the majority result is taken, this bit is correctly identified as a $0$. There are two errors in decoding the second $111$, so this bit is misread as a $0$. It is clear that a character is misread if the decoder makes \emph{two or three errors} in these blocks of three replicates.\\[-0.2cm]
\begin{itemize}
\item[(a)] Show that sending each bit 3 times reduces the error probability from 10\% to 2.8\%. \\ \item[(b)] Using this reduced value, $p=0.028$, calculate the probability that there are no errors in a 20-bit string. Compare this result to Q6(a).  \item[(c)] Now assume that each bit is sent 5 times and, again, the majority vote approach is used. Calculate the probability that there are no errors in a 20-bit string in this case. % \item[(d)] Recalculate the two probabilities from part (c) using the Poisson approximation.
\end{itemize}



\subsection*{Question 7}
Customers arrive to a shop at a rate of 40 per hour and typically stay 30 minutes. \\[-0.2cm]
\begin{itemize}
\item[(a)] How many customers are in the shop on average?  \item[(b)] Through advertising, the shop can increase the arrival rate to 60 per hour. How many customers are in the shop now (assuming they still spend 30 minutes)?  \item[(c)] The shop is small and is now too full. However, by streamlining the layout we can reduce the average time spent in the shop (without compromising profits). How much would the average time need to drop to in order that with 60 customers arriving per hour, there are still the same number of customers in the shop as in part (a).


\end{itemize}







\end{document} 
