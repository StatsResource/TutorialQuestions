\newpage
Q5. The standard deviation of test scores obtained for a certain exam is 18 points. 
A random sample of 81 students has a sample mean of 70 points.

(a) State the point estimate for the mean score for all the students.
(b) Find the 95\% confidence interval for the average score for all students.
(c) Find the 99\% confidence interval for the average score for all students.

Q6. The amount spent (€’s) by customers in a shop are normally distributed. 
A random sample of 16 customers have these values:
\[19 21 35 29 12 35 7 18 21 14 29 20 12 24 32 23\]
(Sample  mean of €21.94 and a sample standard deviation of €8.40) 
Estimate a 95% confidence interval for the population mean.

Q7. The operating life of rechargeable cordless screwdrivers produced by a firm is assumed to 
the approximately normally distributed. A sample of 15 screwdrivers is tested and the mean 
life is found to be 8900 hours, with a sample standard deviation of 500 hours. 
Provide a 95% confidence interval for the population mean.


Question 1 
An IT competency test, used for staff recruitment, is devised so as to give a normal distribution of scores with a mean of 100. A random sample of 49 experienced IT users who are given the test achieve a mean score of 121 with a standard deviation of 14. 
Compute a 95% confidence interval for the group.

\newpage

Question 5
A manufacturer of computer monitors  has, for many years, used a process giving a mean  working life of 4700 hours  for components.
A new process is tried to see if it will increase the life significantly. A sample of 100 monitors gave a mean life of 5000 hours, with a standard deviation of 1400 hours.
i.	Compute a 95% confidence interval for the mean life of components built using the new process.



\newpage

MA4413 2013 Tutorial for Week 8 : Confidence intervals and Hypothesis testing
Question 1. 
An IT competency test, used for staff recruitment, is devised so as to give a normal distribution of scores with a mean of 100. A random sample of 49 experienced IT users who are given the test achieve a mean score of 121 with a standard deviation of 14. 
i.	Perform a hypothesis test to assess whether all such experienced IT Users is unusual (i.e. have a different mean from the general population).
ii.	Compute a 95% confidence interval for the group.

Question 2.
A claim has been made that the mean body temperature of healthy adults is equal to 98.6 degrees. Test this hypothesis using a 0.05 level significance, given the following information.
A sample of 121 people has produced a mean body temperature of 98.2 degrees and a standard deviation of 6.6 degrees.  
Question 3.
The quality control manager at the Telektronic Company considers the production of telephone answering machines to be ’out of control’ when the overall rate of defects exceeds 6%. 
Testing of a random sample of 150 machines revealed that 12 are defective. The production manager claims that production is not out of control and no corrective action is necessary.
i.	Compute a 95% confidence interval for the rate of defective components
ii.	Use a 0.05 significance level to test the production manager’s claim.
(N.B. The Standard Error used in hypothesis testing is different to the one used for confidence intervals)
Question 4.
A manufacturer of computer monitors  has, for many years, used a process giving a mean  working life of 4700 hours  for components.
A new process is tried to see if it will increase the life significantly. A sample of 100 monitors gave a mean life of 5000 hours, with a standard deviation of 1400 hours.
i.	Compute a 95% confidence interval for the mean life of components built using the new process.
ii.	Does this new process make a difference at the 5% level of significance? (Perform a two tailed test, then a one tailed test)




Question 5.
In a study of store checkout scanners, 240 items were checked and 6 of those items were found to be “overcharges”.
Use a 0.05 significance level to test the claim that with these scanners, 1.5 % of sales transactions are overcharges.
(N.B. The Standard Error used in hypothesis testing is different to the one used for confidence intervals)

	\item \textbf{Worked Example 3} \\ Ten replicate analyses of the concentration
	of mercury in a sample of commercial gas condensate gave the
	following results (in ng/ml) :
	
	\begin{tabular}{|c|c|c|c|c|c|c|c|c|c|}
		\hline
		23.3 & 22.5 & 21.9 & 21.5 & 19.9 & 21.3 & 21.7 & 23.8 & 22.6 &
		24.7\\
		\hline
	\end{tabular}
	
	\item \textbf{Wasking Machine Example}%example: 
	Suppose a machine produces circular parts for an electrical component. A random sample of 10 circular parts is selected and the variance of the sample is found to be 0.7 centimetres. The machine is producing the correct standard of circular parts if the variance of the diameter is no larger than 0.4cm.
	
	Suppose that the sampled measurements are normally distributed. A hypothesis test may be carried out to determine whether or not the machine is producing the correct standard of part.
	
	\begin{description}
		\item[$H_0$] $s^2=0.4$
		\item[$H_1$] $s^2>0.4$
	\end{description} 
	
	\begin{itemize}
		\item At a 5\% significance level, the critical value for the test is ? 2 9 ˜ 16.92 . 
		\item The test statistic is ? 2 = n - 1 s 2 s 2 
		\item H 0 = 10 - 1 × 0.7 0.4 = 15.75 .
	\end{itemize}
	
	
	
	
	As the test statistic is less than the critical value, there is no evidence to reject the null hypothesis. The machine is therefore producing a sufficient standard of circular parts.
	
	\item \textbf{Question 42. - Sample Size Estimation } % 10 Marks
	Let $\pi$ be the proportion of workers in Ireland who spend at least one hour
	per day in front of a computer terminal. Suppose that a researcher is going to take a
	sample of n workers and estimate $\pi$ using $\hat{p}$, the proportion of workers in the sample
	who spend at least one hour per day in front of a computer terminal.
	
	\begin{itemize}
		\item[a.] (1 mark) How large
		should $n$ be if the researcher wants to be 90\% certain that his error is less than 0.01?
	\end{itemize}
\end{enumerate}		

\end{document}