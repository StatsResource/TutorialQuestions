
\newpage
\section{KB Tutorial 6}


\subsection*{Question 2}
Consider an $M/M/1$ system with arrivals $X_a \sim \text{Poisson}(\lambda_a=3 \text{ / minute})$ and service time \\$T_s \sim \text{Exponential}(\lambda_s=4 \text{ / minute})$. Calculate the following:\\[-0.2cm]

{\bf(a)} The expected time spent in the system. \quad {\bf(b)} The expected time spent in the queue component. \quad {\bf(c)} The expected number of individuals in the system.  \quad {\bf(d)} The expected number of individuals in the queue component. \quad {\bf(e)} The utilisation factor. \quad {\bf(f)} The probability that an individual spends more than 2 minutes in the system. \quad {\bf(g)} The probability that less than 3 individuals exit the system in a 1 minute period.





\subsection*{Question 3}
Customers arrive to a deli counter at a rate of 12 per hour. On average it takes 3 minutes to serve a customer at this counter. Customers then exit and head to another counter to pay. It takes 1 minute to deal with a customer at this counter. We will assume that arrivals have a Poisson$(\lambda_a)$ distribution and service times have Exponential$(\lambda_{s1})$ and Exponential$(\lambda_{s2})$ distributions respectively (hint: this is a sequence of two $M/M/1$ systems).\\[-0.2cm]

{\bf(a)} What is the average time spent in each sub-system? \quad {\bf(b)} What is the average total time spent in the system? \quad {\bf(c)} How many customers are there (on average) in the system?  \quad \\{\bf(d)} Calculate the utilisation factor for each sub-system. \quad {\bf(e)} What is the average total queueing time? (i.e., total time excluding service time) \quad {\bf(f)} Calculate the probability that at least 20 people exit the shop (i.e., the whole system) in one hour.



\subsection*{Question 4}
{\footnotesize({\bf Note}: this is not a queueing theory question. It is a generalisation of a question which appears on Tutorial2)}\\[0.1cm]
There are two possible routes to a particular location. You take $R_1$ 80\% of the time and $R_2$ 20\% of the time. We assume that travel time has an exponential distribution and, furthermore, the average travel time is 0.25 hours if you take $R_1$ and 0.5 hours if you take $R_2$.\\[-0.2cm]

{\bf(a)} Calculate the probability that the journey takes more than 0.5 hours for each of the routes, i.e., $\Pr(T > 0.5\,|\,R_1)$ and $\Pr(T > 0.5\,|\,R_2)$ respectively. \quad {\bf(b)} Calculate $\Pr(T > 0.5)$. (hint: law of total probability) \quad {\bf(c)} Given that $T>0.5$ hours, what is the probability that you used $R_1$? (i.e., calculate $\Pr(R_1\,|\,T>0.5)$) \quad {\bf(d)} Derive a general expression for $\Pr(R_1\,|\,T>t)$ and evaluate it at $t=0.25$, $t = 1$ and $t = 2$ respectively. Interpret the results.








\end{document}

