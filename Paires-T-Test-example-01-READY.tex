
\begin{frame}
\subsection*{Example 2: Paired Difference (a)}
\begin{itemize}
\item An automobile manufacturer collects mileage data for a sample of $n = 10$ cars in various weight categories
using a standard grade of gasoline with and without a particular additive. \item Of course, the engines were tuned to the same
specifications before each run, and the same drivers were used for the two gasoline conditions (with the driver in fact being
unaware of which gasoline was being used on a particular run). \item Given the mileage data on the next slide,  test the hypothesis
that there is no difference between the mean mileage obtained with and without the additive, using the 5 percent level of
significance \end{itemize}
\end{frame}
%-------------------------------------------------------------------------------------------%
\begin{frame}
\subsection*{Example 2: Paired Difference (b)}
\small
\begin{center}
\begin{tabular}{|c|c|c|c|c|}\hline
car & with additive & without additive & $d_i$ & $d^2_i$\\\hline
1&36.7&36.2&0.5&0.25\\\hline
2&35.8&35.7&0.1&0.01\\\hline
3&31.9&32.3&-0.4&0.16\\\hline
4&29.3&29.6&-0.3&0.09\\\hline
5&28.4&28.1&0.3&0.09\\\hline
6&25.7&25.8&-0.1&0.01\\\hline
7&24.2&23.9&0.3&0.09\\\hline
8&22.6&22.0&0.6&0.36\\\hline
9&21.9&21.5&0.4&0.16\\\hline
10&20.3&20.0&0.3&0.09\\\hline
\end{tabular}
\end{center}
\end{frame}

%--------------------------------------------------------------------------------------------------------------------------%
%-------------------------------------------------------------------------------------------%
\begin{frame}
\subsection*{Example 2: Paired Difference (c)}
\begin{itemize}
\item The average of the case wise differences is computed as \[\bar{d} = {\sum d_i \over n}\]
\[ \bar{d} = { 0.05 + 0.1  - 0.4 + \ldots + 0.30 \over 10 }= 0.17 \]
\item Also, using last column, $\sum d^2_i = (0.25 + 0.01 + 0.16 + \ldots + 0.09) = 1.31$
\end{itemize}

\end{frame}


\begin{frame}
\subsection*{Example 2: Paired Difference (d)}
\textbf{Sample standard deviation of the case-wise differences}:
\large
\[s_d = \sqrt{ {\sum d_i^2 - n\bar{d}^2 \over n-1}}\]
We know the following:
\begin{itemize}
\item The sample size $n$ which is 10.
\item The average of the case-wise differences. $\bar{d} = 0.17$
\item  $\sum d^2_i = 1.31$
\end{itemize}
\end{frame}



\begin{frame}
\subsection*{Example 2: Paired Difference (e)}
\textbf{Sample standard deviation  of the case-wise differences}://
\[s_d = \sqrt{ {\sum d_i^2 - n\bar{d}^2 \over n-1}}\]

\[s_d = \sqrt{ { 1.31 - 10(0.17)^2 \over 9}} = 0.337\]

\textbf{The standard error:} \[ S.E.(\bar{d}) = s_d / \sqrt{n} = {0.0337 \over 3.16} = 0.107\]
\end{frame}

\begin{frame}
\subsection*{Example 2: Paired Difference (f)}
\textbf{Null and Alternative Hypotheses}:
\begin{itemize}
\item That is, the null hypothesis is:\\
$H_0: \mu_d = 0$ Additive makes no difference to performance\\
$H_1: \mu_d \neq 0$ Additive makes a significant difference to performance \\
\end{itemize}
\textbf{Test Statistic}:
\begin{itemize}
\item TS = 0.17 / 0.107 = 1.59
\end{itemize}
\end{frame}

\begin{frame}
\subsection*{Example 2: Paired Difference (g)}
\textbf{Critical value}:
\begin{itemize}
\item $\alpha = 0.05, k = 2$ \item small sample , so $df = n-1 = 9$
\item As with an earlier example, CV is computed as follows \texttt{ qt(0.975,df=9) =2.262}
\end{itemize}
\bigskip
\textbf{Decision Rule}:\\
Is $|TS| > CV$? No, we fail to reject the null hypothesis.

\subsection*{Example 3: Difference in Means (a) }
Two sets of patients are given courses of treatment under two different drugs. The benefits
derived from each drug can be stated numerically in terms of the recovery times; the readings are given below.

\begin{itemize}
\item Drug 1:  $n_1$ = 40 , $\bar{x}_1$ = 3.3 days and $s_1 = 1.524$
\item Drug 2:  $n_2$ = 45 , $\bar{x}_2$ = 4.3 days and $s_2 = 1.951 $
\end{itemize}
\end{frame}

%-------------------------------------------------------------------------------------------%
\begin{frame}
\subsection*{Example 3: Difference in Means (b) }
\begin{itemize}
\item
The first step in hypothesis testing is to specify the null hypothesis and an alternative hypothesis.
\item When testing differences between mean recovery times, the null hypothesis is that the two population means are equal.
\item That is, the null hypothesis is:\\
$H_0: \mu_1 = \mu_2$ ( The population means are equal)\\
$H_1: \mu_1 \neq \mu_2$ (The population means are different)\\
\end{itemize}
(Remark: Two Tailed Test, therefore $k = 2$, and $\alpha = 0.05$)


%-------------------------------------------------------------------------------------------%

\subsection*{Example 3: Difference in Means (c) }
\begin{itemize}
\item The observed difference in means is 1 day.
\item The relevant formula for the standard error is
\[ S.E(\bar{x}_1 - \bar{x}_2) = \sqrt{{s^2_1\over n_1}+{s^2_2 \over n_2}} \]
\item  \[ S.E(\bar{x}_1 - \bar{x}_2) = \sqrt{{(1.524)^2 \over 40}+{(1.951)^2 \over 45}}   \]
\item  \[ S.E(\bar{x}_1 - \bar{x}_2) = 0.377\mbox{ days}\]
\end{itemize}


%-------------------------------------------------------------------------------------------%
\begin{frame}[fragile]
\subsection*{Example 2: Difference in Means (d) }
\begin{itemize}
\item The Test statistic is therefore
\[ TS = {\mbox{observed} - \mbox{null} \over \mbox{Std. Error}}  = {1 - 0 \over 0.377 } = 2.65 \]
\item Lets compute the p-value of this : \\
p-value = $P(z \geq 2.65) = 0.0040$
\begin{verbatim}
> 1-pnorm(2.65)
[1] 0.004024589
\end{verbatim}

\item What is this value smaller than threshold $\alpha / k$? \\
\item $\alpha / k$ = $0.05/2$ = 0.0250? Yes the p-value is smaller than this.
\item \textbf{Conclusion:} we reject the null hypothesis. There is a significant different between both drugs, in terms of recovery times.

\end{itemize}

