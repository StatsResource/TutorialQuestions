
\textbf{Example}

\begin{itemize}
\item The standard deviation of the life for a particular brand of ultraviolet tube is known to be $s = 500$ hr,
\item Also it is assumed, but not known, that the operating life of the tubes is normally distributed. \item The manufacturer claims that average tube life
is at least 9,000hr. \item Test this claim at the 5 percent level of significance against the alternative hypothesis
that the mean life is less than 9,000 hr, and given that for a sample of $n = 10$ tubes the mean operating
life was $\bar{x}  =  8,800$ hr.
\end{itemize}

\textbf{Example}

\begin{itemize}
\item $H_0 \mbox{ : } $ $\mu \geq 9000$Average life span is not less than 9000 hours.
\item $H_1 \mbox{ : } $ $\mu < 9000$    Average life span is  less than 9000 hours.
\end{itemize}
\bigskip
\begin{itemize}
\item The observed difference is -200 hours. (i.e. 8,800 - 9,000 hours)
\item The standard error is determined from formulae.
\[ S.E. \bar{x}  = {s \over \sqrt{n}} = {500 \over \sqrt{10}} \]
\end{itemize}

\textbf{Example}

\begin{itemize}
\item The CV is determined from Murdoch Barnes Table 7, with $\alpha = 0.05$ and $k = 1$.
\item The sample is small n= 10  $df = n-1 = 9$.Therefore CV = 
\item (Remark: If the distribution was known to be normal, we could use $df = \infty$, i.e $CV = 1.645$).
\item Decision rule : Is $|TS| >CV$? Yes. 
\end{itemize}




\section{Example}


\subsection{Example 1}
A survey of study habits wishes to determine whether the mean
study hours completed by women at a particular college is higher
than for men at the same college. A sample of $n_1$ = 10 women and
$n_2$ = 12 men were taken, with mean hours of study $\bar{x}_1$ =
120 and $\bar{x}_2$ = 105 respectively. The standard deviations
were known to be $\sigma_1$ = 28 and $\sigma_2$ = 35.

The hypothesis being tested is:

\begin{eqnarray}
H_{0}: \mu_1 = \mu_2\qquad \qquad (\mu_1 - \mu_2= 0)\\
H_{a}: \mu_1 \neq \mu_2 \qquad \qquad (\mu_1 - \mu_2 \neq 0)
\end{eqnarray}

In $R$, the test statistic is calculated using:

\begin{verbatim}
xbar1 <- 120
xbar2 <- 105
sd1 <- 28
sd2 <- 35
n1 <- 10
n2 <-12
TS <- ( (xbar1 - xbar2) - (0) )/sqrt( (sd1^2/n1) + (sd2^2/n2) )
TS
[1] 1.116536
\end{verbatim}
Now need to calculate the critical value or the p-value.


The critical value can be looked up using qnorm. Since this is a
one-tailed test and there is a > sign in $H_1$:

\begin{verbatim}
qnorm(0.95)
[1] 1.644854
\end{verbatim}

Since the test statistic is less than the critical value ( 1.116536 < 1:645 )there is not enough evidence to reject $H_0$
and conclude that the population mean hours study for women is
not higher than the population mean hours study for men.



The p-value is determined using pnorm.

Careful! Remember pnorm
gives the probability of getting a value LESS than the value specified. We want the probability of getting a value greater than
the test statistic.
\begin{verbatim}
1-pnorm(1.116536) # OR pnorm(1.116536, lower.tail=FALSE)
[1] 0.1320964
\end{verbatim}






