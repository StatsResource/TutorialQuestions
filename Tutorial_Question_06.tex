
%----------------------------------------------------%
	
	\section{Worked Examples : Repeat 2006}
	Suppose an oil exploration company purchases drill bits that have a life span that is approximately normally distributed, with a mean equal to 80 hours and a standard deviation equal to 10 hours.
	
	\begin{itemize}
		\item[(i)]	What is the probability that a drill bit will fail before 60 hours of use?
		
		\item[(ii)]	What is the probability that a drill bit will last between 70 hours and 90 hours?
		
		\item[(iii)]	The life span of 95\% of drill bits is below what value?
		
	\end{itemize}
	
	
	\section{Worked Examples : MA4104 Business Statistics SPRING 2008}
	
	A tyre manufacturer claims that under normal driving conditions, the tread life of a certain tyre follows a normal distribution with mean 50,000 miles and standard deviation 5000 miles. 
	
	\begin{itemize}
		\item[(i)] If your tyres wear out at 45,000 miles, would you consider this unusual? Support your answer with an appropriate probability calculation using the normal curve. [ 10 marks ] 
		
		\item[(ii)] If the manufacturer sells 100,000 of these tyres and warrants them to last at least 40,000 miles, about how many tyres will wear out before the warranty expires? [ 10 marks ] 
	\end{itemize}
\section{Worked Example 3 - with Solutions}
Assume that the number of weekly study hours for students at a certain university
is approximately normally distributed with a mean of 22 and a standard deviation
of 6.
\begin{enumerate}
	\item Find the probability that a randomly chosen student studies less than 12
	hours.
	\item Estimate the percentage of students that study more than 37 hours.
\end{enumerate}

\textbf{solution}
$X \sim \mathcal(22,6^2)$  ( in form $X \sim \mathcal(\mu,\sigma^2)$\\

Part 1: $P(X \leq 12)$\\


$Z_1 = \frac{12 - 22}{6} = \frac{-10}{6} = -1.66 $\\

Part 2: $P(X \geq 37)$\\

$Z_2 = \frac{37 - 22}{6} = \frac{15}{6} = 2.5 $\\


\section{Worked Example 4 - with Solutions}
The mean is 550kg, with standard deviation 150kg, and we are interested in the area that is greater than 600kg.

\begin{equation}
Z = \frac{ X - \mu }{ \sigma }
\end{equation}

Here X = 600kg,
$\mu$ , the mean = 550kg
$\sigma$, the standard deviation = 150kg
\begin{itemize}
	\item $z = ( 600 - 550 ) / 150$
	\item $z = 50 / 150$
	\item $z = 0.33$
\end{itemize}

Look in the table down the left hand column for z = 0.3, and across under 0.03.
The number in the table is the tail area for z=0.33 which is 0.3707.
This is the probability that the weight will exceed 600kg.



%----------------------------------------------------%

\section{Worked Example 8 - with Solutions (Tyres)}
% \emph{Taken from MA4104 Business Statistics Examination paper, Spring 2008}\\
% Q1. (a) 
A tyre manufacturer claims that under normal driving conditions, the tread life of a certain tyre follows a normal distribution with mean 50,000 miles and standard deviation 5000 miles.

\begin{itemize}
	\item[(i)] If your tyres wear out at 45,000 miles, would you consider this unusual? Support your answer with an appropriate probability calculation using the normal curve. [ 10 marks ]
	\item[(ii)] If the manufacturer sells 100,000 of these tyres and warrants them to last at least 40,000 miles, about how many tyres will wear out before the warranty expires? [ 10 marks ]
\end{itemize}


Part (i) Solution

\begin{itemize}
	\item Test Value ; 45,000km			
	\item Mean		 km	
	\item Standard Deviation	 km
\end{itemize}


Find  

Standardisation
Apply the standardisation formula	 	to test value


i.e.  = 

To find   we use the “Property of Symmetry”

“Property of Symmetry” -   for any value A

From Murdoch Barnes Tables (page 13)  

Therefore   = 0.1587 

“Complement Rule”		 =1-  for any given value A

=1-  = 0.8413

Conclusion

15.87\% of Tyres are expected to last less than 45,000km

\begin{itemize}
\item 84.13\% of Tyres are expected to last longer than 45,000km




Part (i) Solution

Lower Limit ; 40,000km			Mean		 km	
Standard Deviation	 km

Find  

\item Standardisation
Apply the standardisation formula	 	to limit


i.e.  = 

\item To find   we use the “Property of Symmetry”

“Property of Symmetry” -   for any value A


\item From Murdoch Barnes Tables (page 13)  

Therefore   = 0.02275 

\item \textbf{Conclusion}
2.275\% of Tyres are expected to last less than 40,000km

Of a Batch of 100,000 tyres,  2270 tyres will wear out before the warranty expires.
\end{itemize}
%==================================================%


\section{Worked Example 10 - with Solutions}
%- Spring 2006 Q3.    (a)	

The breaking strength of a certain type of plastic block is normally distributed with a
mean of 1500kg and standard deviation of 50kg. 

\begin{itemize}
	\item (iv)	What is the probability that a block with have a breaking strength greater than 1570kg?
	\item (v)	What is the probability that a block with have a breaking strength measuring between 1482kg and 1518kg?
	\item (iii)	Determine the maximum load such that no more than 5`\% of the blocks break?
\end{itemize}
